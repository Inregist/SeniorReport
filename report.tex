%%%%%% Run at command line, run
%%%%%% xelatex grad-sample.tex 
%%%%%% for a few times to generate the output pdf file
\documentclass[12pt,oneside,openright,a4paper]{cpe-thai-project}


\usepackage{polyglossia}
\setdefaultlanguage{thai}
\setotherlanguage{english}
\newfontfamily\thaifont[Script=Thai,Scale=1.23]{TH Sarabun New}
\defaultfontfeatures{Mapping=tex-text,Scale=1.23,LetterSpace=0.0}
\setmainfont[Scale=1.23,LetterSpace=0,WordSpace=1.0,FakeStretch=1.0,Mapping=tex-text]{TH Sarabun New}
\XeTeXlinebreaklocale "th"	
\XeTeXlinebreakskip = 0pt plus 0pt
\emergencystretch=10pt

\usepackage{float}
\usepackage{multirow}
\usepackage[table,xcdraw]{xcolor}
%%%%%%%%%%%%%%%%%%%%%%%%%%%%%%%%%%%%%%%%%%%%%%%%%%%%%%%%%%%%%%%%%%%
% Customize below to suit your needs 
% The ones that are optional can be left blank. 
%%%%%%%%%%%%%%%%%%%%%%%%%%%%%%%%%%%%%%%%%%%%%%%%%%%%%%%%%%%%%%%%%%%
% First line of title
\def\disstitleone{Sweets vs Sweets :}   
% Second line of title
\def\disstitletwo{Real Time Strategy Multiplayer Tower Defense game}   
% Your first name and lastname
\def\dissauthor{Mr. Chirateep Pakdeengam}   % 1st member
%%% Put other group member names here ..
\def\dissauthortwo{Ms. Premyuda Angkawichai}   % 2nd member (optional)
\def\dissauthorthree{}   % 3rd member (optional)

% The degree that you're persuing..
\def\dissdegree{Bachelor of Engineering} % Name of the degree
\def\dissdegreeabrev{B.Eng} % Abbreviation of the degree
\def\dissyear{2021}                   % Year of submission
\def\thaidissyear{2564}               % Year of submission (B.E.)

%%%%%%%%%%%%%%%%%%%%%%%%%%%%%%%%%%%%%%%%%%%%
% Your project and independent study committee..
%%%%%%%%%%%%%%%%%%%%%%%%%%%%%%%%%%%%%%%%%%%%
\def\dissadvisor{Assoc.Prof. Dr.Natasha Dejdumrong, D.Tech.Sci.}  % Advisor
%%% Leave it empty if you have no Co-advisor
\def\disscoadvisor{}  % Co-advisor
\def\disscommitteetwo{Asst.Prof. Dr.Nuttanart Facundes, Ph.D.}  % 3rd committee member (optional)
\def\disscommitteethree{Asst.Prof. Dr.Khajonpong Akkarajitsakul, Ph.D.}   % 4th committee member (optional) 
\def\disscommitteefour{Assoc.Prof. Dr.Naruemon Wattanapongsakorn, Ph.D.}    % 5th committee member (optional) 

\def\worktype{Project} %%  Project or Independent study
\def\disscredit{3}   %% 3 credits or 6 credits

\def\fieldofstudy{Computer Engineering} 
\def\department{Computer Engineering} 
\def\faculty{Engineering}

\def\thaifieldofstudy{วิศวกรรมคอมพิวเตอร์} 
\def\thaidepartment{วิศวกรรมคอมพิวเตอร์} 
\def\thaifaculty{วิศวกรรมศาสตร์}
 
\def\appendixnames{Appendix} %%% Appendices or Appendix

\def\thaiworktype{ปริญญานิพนธ์} %  Project or research project % 
\def\thaidisstitleone{Sweets vs Sweets :}
\def\thaidisstitletwo{Real Time Strategy Multiplayer Tower Defense game}
\def\thaidissauthor{นายจิรทีปต์ ภักดีงาม}
\def\thaidissauthortwo{นางสาวเปรมยุดา อังคะวิชัย} %Optional
\def\thaidissauthorthree{} %Optional

\def\thaidissadvisor{รศ.ดร.ณัฐชา เดชดำรง}
%% Leave this empty if you have no co-advisor
\def\thaidisscoadvisor{} %Optional
\def\thaidissdegree{วิศวกรรมศาสตรบัณฑิต}

% Change the line spacing here...
\linespread{1.15}

%%%%%%%%%%%%%%%%%%%%%%%%%%%%%%%%%%%%%%%%%%%%%%%%%%%%%%%%%%%%%%%%
% End of personal customization.  Do not modify from this part 
% to \begin{document} unless you know what you are doing...
%%%%%%%%%%%%%%%%%%%%%%%%%%%%%%%%%%%%%%%%%%%%%%%%%%%%%%%%%%%%%%%%


%%%%%%%%%%%% Dissertation style %%%%%%%%%%%
%\linespread{1.6} % Double-spaced  
%%\oddsidemargin    0.5in
%%\evensidemargin   0.5in
%%%%%%%%%%%%%%%%%%%%%%%%%%%%%%%%%%%%%%%%%%%
%\renewcommand{\subfigtopskip}{10pt}
%\renewcommand{\subfigbottomskip}{-5pt} 
%\renewcommand{\subfigcapskip}{-6pt} %vertical space between caption
%                                    %and figure.
%\renewcommand{\subfigcapmargin}{0pt}

\renewcommand{\topfraction}{0.85}
\renewcommand{\textfraction}{0.1}

\newtheorem{theorem}{Theorem}
\newtheorem{lemma}{Lemma}
\newtheorem{corollary}{Corollary}

\def\QED{\mbox{\rule[0pt]{1.5ex}{1.5ex}}}
\def\proof{\noindent\hspace{2em}{\itshape Proof: }}
\def\endproof{\hspace*{\fill}~\QED\par\endtrivlist\unskip}
%\newenvironment{proof}{{\sc Proof:}}{~\hfill \blacksquare}
%% The hyperref package redefines the \appendix. This one 
%% is from the dissertation.cls
%\def\appendix#1{\iffirstappendix \appendixcover \firstappendixfalse \fi \chapter{#1}}
%\renewcommand{\arraystretch}{0.8}
%%%%%%%%%%%%%%%%%%%%%%%%%%%%%%%%%%%%%%%%%%%%%%%%%%%%%%%%%%%%%%%%
%%%%%%%%%%%%%%%%%%%%%%%%%%%%%%%%%%%%%%%%%%%%%%%%%%%%%%%%%%%%%%%%

\usepackage{ragged2e}
\begin{document}

\pdfstringdefDisableCommands{%
\let\MakeUppercase\relax
}

\begin{center}
  \includegraphics[width=2.8cm]{imgs/logo02.jpg}
\end{center}
\vspace*{-1cm}

\maketitlepage
\makesignaturepage 

%%%%%%%%%%%%%%%%%%%%%%%%%%%%%%%%%%%%%%%%%%%%%%%%%%%%%%%%%%%%%%
%%%%%%%%%%%%%%%%%%%%%% English abstract %%%%%%%%%%%%%%%%%%%%%%%
%%%%%%%%%%%%%%%%%%%%%%%%%%%%%%%%%%%%%%%%%%%%%%%%%%%%%%%%%%%%%%

\abstract

Resource management, Planning and Decision making are fundamental skills 
that everyone should learn. But it is too risky to learn these skills in real-life situations.

We would like to create a Real-time multiplayer Tower Defense game for people to practice 
those skills mentioned above with risk-free. Players can challenge others to fight in the 
world of delicious sweets and candy. Our target group are people who love to play games such as 
students between the ages of 15-24 years old. The game can be played on MacOS and Windows PC.

\begin{flushleft}
\begin{tabular*}{\textwidth}{@{}lp{0.8\textwidth}}
\textbf{Keywords}: & Real Time Strategy Game / Tower Defense / Multiplayer Game / 2.5D Isometric Game / Machine Learning
\end{tabular*}
\end{flushleft}
\endabstract

%%%%%%%%%%%%%%%%%%%%%%%%%%%%%%%%%%%%%%%%%%%%%%%%%%%%%%%%%%%%%%
%%%%%%%%%% Thai abstract here %%%%%%%%%%%%%%%%%%%%%%%%%%%%%%%%%
%%%%%%%%%%%%%%%%%%%%%%%%%%%%%%%%%%%%%%%%%%%%%%%%%%%%%%%%%%%%%%
% {\newfontfamily\thaifont{TH Sarabun New:script=thai}[Scale=1.3]
% \XeTeXlinebreaklocale "th_TH"	
% \thaifont

\thaiabstract

การจัดการ การวางแผน และการตัดสินใจเป็นทักษะพื้นฐานที่ทุกคนควรมี 
แต่ถ้าเรียนรู้และฝึกทักษะดังกล่าวในสถานการณ์จริงจะค่อนข้างมีความเสี่ยง เช่น การฝึกวางแผนการลงทุนโดยใช้เงินจริง 
มีโอกาสเสี่ยงสูงในการขาดทุน 

ผู้จัดทำจึงต้องการพัฒนาเกมที่สามารถให้ผู้เล่นได้ฝึกทัศนคติการวางแผน 
ทักษะการตัดสินใจ และการแก้ไขปัญหาเฉพาะหน้าได้โดยปลอดภัย ไร้ความเสี่ยง 
พร้อมสนุกไปกับการเรียนรู้ โดยเป็นเกม Tower Defense ที่มีการต่อสู้กันระหว่างผู้เล่นสองคนแบบ 
real-time multiplayer จำลองเหตุการณ์ในเกมให้อยู่ในโลกของขนมหวานที่น่าดึงดูดใจ 
โดยมีกลุ่มเป้าหมายคือกลุ่มคนที่ชื่นชอบการเล่นเกม ได้แก่ นักเรียนนักศึกษาที่อยู่ในช่วงอายุ 15-24 ปี
สามารถเล่นได้ทั้งบนระบบปฏิบัติการ Windows และ MacOS


\begin{flushleft}
\begin{tabular*}{\textwidth}{@{}lp{0.8\textwidth}}
 & \\

\textbf{คำสำคัญ}: & Real Time Strategy Game / Tower Defense / Multiplayer Game / 2.5D Isometric Game / Machine Learning
\end{tabular*}
\end{flushleft}
\endabstract

%}

%%%%%%%%%%%%%%%%%%%%%%%%%%%%%%%%%%%%%%%%%%%%%%%%%%%%%%%%%%%%
%%%%%%%%%%%%%%%%%%%%%%% Acknowledgments %%%%%%%%%%%%%%%%%%%%
%%%%%%%%%%%%%%%%%%%%%%%%%%%%%%%%%%%%%%%%%%%%%%%%%%%%%%%%%%%%

\preface
โครงงานนี้สำเร็จลงได้ด้วยความช่วยเหลืออย่างดียิ่งจาก รศ.ดร.ณัฐชา เดชดำรง ที่ปรึกษาโครงงาน 
ที่กรุณาสละเวลาให้ความรู้ คำปรึกษา คำแนะนำ และข้อเสนอแนะที่เป็นประโยชน์อย่างมาก 
อีกทั้งยังคอยติดตามดูแลเอาใจใส่ตลอดการทำโครงงานนี้จนสำเร็จลุล่วงได้ด้วยดี 
ผู้จัดทำโครงงานจึงขอกราบขอบพระคุณเป็นอย่างสูงไว้ ณ ที่นี้ด้วย

สุดท้ายนี้ขอขอบคุณเพื่อน ๆ พี่ ๆ และ น้อง ๆ ในภาควิชาวิศวกรรมคอมพิวเตอร์ทุกคนที่คอยให้ความช่วยเหลือเป็นอย่างดี


%%%%%%%%%%%%%%%%%%%%%%%%%%%%%%%%%%%%%%%%%%%%%%%%%%%%%%%%%%%%%
%%%%%%%%%%%%%%%% ToC, List of figures/tables %%%%%%%%%%%%%%%%
%%%%%%%%%%%%%%%%%%%%%%%%%%%%%%%%%%%%%%%%%%%%%%%%%%%%%%%%%%%%%
% The three commands below automatically generate the table 
% of content, list of tables and list of figures
\tableofcontents                    
\listoftables
\listoffigures                      

%%%%%%%%%%%%%%%%%%%%%%%%%%%%%%%%%%%%%%%%%%%%%%%%%%%%%%%%%%%%%%
%%%%%%%%%%%%%%%%%%%%% List of symbols page %%%%%%%%%%%%%%%%%%%
%%%%%%%%%%%%%%%%%%%%%%%%%%%%%%%%%%%%%%%%%%%%%%%%%%%%%%%%%%%%%%
% You have to add this manually..
% \listofsymbols
% \begin{flushleft}
% \begin{tabular}{@{}p{0.07\textwidth}p{0.7\textwidth}p{0.1\textwidth}}
% \textbf{SYMBOL}  & & \textbf{UNIT} \\[0.2cm]
% $\alpha$ & Test variable\hfill & m$^2$ \\
% $\lambda$ & Interarival rate\hfill &  jobs/second\\
% $\mu$ & Service rate\hfill & jobs/second\\
% \end{tabular}
% \end{flushleft}
%%%%%%%%%%%%%%%%%%%%%%%%%%%%%%%%%%%%%%%%%%%%%%%%%%%%%%%%%%%%%%
%%%%%%%%%%%%%%%%%%%%% List of vocabs & terms %%%%%%%%%%%%%%%%%
%%%%%%%%%%%%%%%%%%%%%%%%%%%%%%%%%%%%%%%%%%%%%%%%%%%%%%%%%%%%%%
% You also have to add this manually..
% \listofvocab
% \begin{flushleft}
% \begin{tabular}{@{}p{1in}@{=\extracolsep{0.5in}}l}
% Test &  test \\
% MANET & Mobile Ad Hoc Network 
% \end{tabular}
% \end{flushleft}

%\setlength{\parskip}{1.2mm}

%%%%%%%%%%%%%%%%%%%%%%%%%%%%%%%%%%%%%%%%%%%%%%%%%%%%%%%%%%%%%%%
%%%%%%%%%%%%%%%%%%%%%%%% Main body %%%%%%%%%%%%%%%%%%%%%%%%%%%%
%%%%%%%%%%%%%%%%%%%%%%%%%%%%%%%%%%%%%%%%%%%%%%%%%%%%%%%%%%%%%%%


\chapter{บทนำ}

\section{ที่มาและความสำคัญ}

การจัดการ การวางแผน และการตัดสินใจเป็นทักษะพื้นฐานที่ทุกคนควรมี 
การเรียนรู้ทักษะเหล่านี้ ผู้เรียนรู้ต้องลองผิดลองถูก หาข้อผิดพลาด พัฒนาปรับปรุงวิธีการคิด 
การจัดการ และการตัดสินใจของตนเอง แต่ถ้าเรียนรู้และฝึกทักษะดังกล่าวในสถานการณ์จริง 
จะค่อนข้างมีความเสี่ยง เช่น การฝึกวางแผนการลงทุนโดยใช้เงินจริง มีโอกาสเสี่ยงสูงในการขาดทุน 
ดังนั้น หากมีพื้นที่ทดลองฝึกทักษะดังกล่าวโดยที่ไม่มีความเสี่ยง (Sandbox) ผู้
เรียนจะสามารถเรียนรู้ได้อย่างไม่ต้องเป็นกังวล

การเรียนรู้จากเกมเป็นวิธีการเรียนรู้แบบหนึ่งที่ค่อนข้างมีประสิทธิภาพ และไม่มีความเสี่ยง 
สามารถทำให้ผู้ที่เล่นเกมได้สนุกไปกับการเรียนรู้ หรือได้เรียนรู้บางอย่างจากเกมโดยไม่รู้ตัว 
และเกิดการเรียนรู้ได้ง่าย แต่การออกแบบเกมให้สามารถเรียนรู้ได้อย่างสนุกสนานนั้นไม่ใช่เรื่องง่าย 
ผู้ออกแบบเกมต้องคิดและออกแบบเกมให้ผู้เล่นสามารถสนุกสนาน พร้อมกับแทรกเนื้อหาสาระหรือทักษะบางอย่างเข้าไปในเกม 
โดยไม่ให้ผู้เล่นรู้สึกว่าเนื้อหาแน่นหรือน่าเบื่อเกินไป

ผู้จัดทำอยากพัฒนาเกมที่สามารถให้ผู้เล่นได้ฝึกทัศนคติการวางแผน ทักษะการตัดสินใจ และการแก้ไขปัญหาเฉพาะหน้า 
โดยเป็นการต่อสู้กันระหว่างผู้เล่นสองคนแบบ real-time multiplayer และมีการเก็บข้อมูลการเล่นของผู้เล่น 
เพื่อนำมาพัฒนา AI จาก Machine Learning ที่สามารถเลียนแบบการเล่นให้เหมือนมนุษย์ได้

สาเหตุที่ผู้จัดทำพัฒนาเกมในรูปแบบ real-time multiplayer เนื่องจากการที่ผู้เล่นได้สู้กับผู้เล่นจริงนั้น
ทำให้สิ่งที่เกิดขึ้นภายในเกมมีความสดใหม่ ไม่ซ้ำเดิม เพิ่ม Replayability (คุณค่าของการเล่นเกมเดิมซ้ำ) 
เพราะผู้เล่นแต่ละคนมีความคิดหรือการวางแผนที่แตกต่างกัน และการสร้าง AI จาก Machine Learning 
สามารถนำมาเพื่อวัดระดับของผู้เล่นได้ด้วยการปรับระดับความยากของ AI ทำให้เราสามารถประเมินระดับการเล่นของผู้เล่นได้ 
เพื่อเป็นตัวชี้วัดว่าผู้เล่นสามารถพัฒนาทักษะที่กล่าวไปข้างต้นได้ผ่านการเล่นเกม


\section{วัตถุประสงค์}

\begin{itemize}
\item เพื่อให้ผู้เล่นได้ฝึกกระบวนการคิดวางแผนแบบ Real Time 
\item เพื่อให้ผู้เล่นได้ฝึกทักษะการแก้ไขปัญหาเฉพาะหน้า
\item เพื่อศึกษาและพัฒนาเกมด้วย Unity โดยใช้ภาษา C\#
\end{itemize}


\section{ขอบเขตของโครงงาน}

\begin{itemize}
\item  เกม Tower Defense ที่ผสมผสานกับ Real Time Strategy
\item  สามารถเล่นกับผู้เล่นอื่นได้ผ่านระบบ Multiplayer 
\item  สามารถเล่นได้ผ่านระบบปฏิบัติการ Windows และ MacOS
\item  พัฒนาเกมโดยใช้ Unity Engine และ C\# Programming Language 
\item  สร้าง AI เพื่อเลียนแบบการเล่นของผู้เล่น 
\end{itemize}


\section{ประโยชน์ที่คาดว่าจะได้รับ}

เกมที่ผู้เล่นเล่นได้อย่างสนุกสนานได้ฝึกทักษะการวางแผน การบริหารจัดการ และการตัดสินใจ

\pagebreak
\section{ขั้นตอนการดำเนินงาน}
\textbf{ภาคการศึกษาที่ 1}
\begin{table}[H]
  \caption{แผนการดำเนินงานภาคเรียนที่ 1}\label{tbl:tab-1}
  \includegraphics[width=1.0\textwidth]{./imgs/tbl-1.pdf}
\end{table}

\textbf{ภาคการศึกษาที่ 2}
\begin{table}[H]
  \caption{แผนการดำเนินงานภาคเรียนที่ 2}\label{tbl:tab-2}
  \includegraphics[width=1.0\textwidth]{./imgs/tbl-2.pdf}
\end{table}

\pagebreak
\section{ผลการดำเนินงาน}
\textbf{ภาคการศึกษาที่ 1}
\begin{enumerate}
  \item ข้อเสนอโครงงาน
  \item เอกสารโครงร่างการออกแบบเกม (Game Design Document) 
  \item วิดีโอตัวอย่าง Gameplay
  \item Prototype เกม
  
\end{enumerate}

\textbf{ภาคการศึกษาที่ 2}
\begin{enumerate}
  \item เกมที่สามารถเล่นได้ตามที่ออกแบบไว้
  \item รายงานฉบับสมบูรณ์
  \item User Manual
\end{enumerate}

%%%%%%%%%%%%%%%%%%%%%%%%%%%%%%%%%%%%%%%%%%%%%%%%%%%%%%%%%%%%
%%%%%%%%%%%%%%  Literature Review %%%%%%%%%%%%%%%%%%%%%%%%%%
%%%%%%%%%%%%%%%%%%%%%%%%%%%%%%%%%%%%%%%%%%%%%%%%%%%%%%%%%%%%
\chapter{ทฤษฎีความรู้และงานที่เกี่ยวข้อง}

\section{บทนำ}
ในบทนี้ผู้จัดทำได้ศึกษาและหาข้อมูลเกี่ยวกับทฤษฎีที่เกี่ยวข้อง 
ภาษาหรือเครื่องมือที่จำเป็น และตัวอย่างเกมที่มีความคล้ายคลึงกับสิ่งที่จะศึกษา 
พร้อมทั้งพิจารณาข้อดีข้อเสีย และนำมาปรับใช้ สร้างความแตกต่างให้กับเกมที่จะพัฒนา


\section{ที่มาและความสำคัญ}

การจัดการ การวางแผน และการตัดสินใจเป็นทักษะพื้นฐานที่จำเป็น 
แต่การเรียนรู้ทักษะดังกล่าวต้องสามารถฝึกและเรียนรู้จากการลองผิดลองถูกได้โดยไม่มีความเสี่ยง 
การเรียนรู้จากเกมจึงเป็นวิธีการเรียนรู้ที่สามารถแก้ไขปัญหาดังกล่าวได้ 
โดยผู้เล่นสามารถเรียนรู้จากเกมที่จำลองออกมาเป็นสถานการณ์ให้ได้ฝึกคิดวางแผน 
แก้ปัญหาและตัดสินใจ พร้อมทั้งได้รับความสนุกในระหว่างการเรียนรู้ 
อีกทั้งยังสามารถแข่งขันกับผู้เล่นคนอื่นได้แบบ Real-time เพื่อเจอกับสถานการณ์ใหม่ที่ไม่ซ้ำเดิม 
และเพิ่มคุณค่าของการเล่นเกมเดิมซ้ำ (Replayability)

นอกจากนี้เพื่อให้สามารถวัดระดับทักษะของผู้เล่น ผู้จัดทำวางแผนพัฒนา AI ด้วย Machine Learning 
ที่สามารถเลียนแบบวิธีการเล่นของผู้เล่น และปรับระดับความยากของ AI 
เพื่อเป็นตัวชี้วัดว่าผู้เล่นสามารถพัฒนาทักษะที่กล่าวไปข้างต้นได้ผ่านการเล่นเกม


\section{ทฤษฎีที่เกี่ยวข้อง}

\subsection{Natural Funativity \cite{natfun05}}
Natural Funativity เป็นทฤษฎีที่มีพื้นฐานมาจากแนวคิดว่าความสนุกต่าง ๆ 
ของมนุษย์มีรากฐานมาจากการล่าและเก็บสะสมสิ่งของตั้งแต่สมัยชนเผ่าในอดีต 
โดยความสนุกนั้นได้พัฒนามาเป็นความสนุกจากการฝึกฝนทักษะการเอาตัวรอด
และทักษะทางสังคมในปัจจุบัน สามารถแบ่งออกเป็น 3 ด้าน ดังนี้

\begin{enumerate}
  \item \textbf{Physical Fun}
  
  ความสนุกด้านการใช้ร่างกาย เช่น ความสนุกจากการเล่นกีฬา 
  ความสนุกจากการสำรวจ ความสนุกจากใช้สายตาและมือร่วมกัน เป็นต้น
  
  \item \textbf{Social Fun}
  
  ความสนุกในการเข้าสังคม คือความสนุกที่มาจากการมีปฏิสัมพันธ์กับผู้อื่น 
  การสื่อสาร การแข่งขัน เช่น ความสนุกจากการเล่าเรื่องราว 
  ความสนุกจากการเล่นเกมออนไลน์
  
  \item \textbf{Mental Fun}
  
  ความสนุกทางด้านจิตใจ ไม่ว่าจะเป็นขณะที่คิดวางแผน 
  หรือสามารถทำได้ตามแผนจนชนะ ทำให้ผู้เล่นรู้สึกดีต่อตนเอง 
  เกิดความมั่นใจ และรู้สึกสนุก
\end{enumerate}

\pagebreak
\subsection{Maslow’s Hierarchy of Needs\cite{maslow15}}

\begin{figure}[!h]\centering
\includegraphics[width=10cm]{./imgs/2-1.png}
\caption{ลำดับขั้นความต้องการของมาสโลว์}\label{fig:2-1}
\small [ที่มา : \url{https://droidinterface.com/pages/people_helping_animals_top_scale_human_needs.php}]
\end{figure}

แอบราฮัม มาสโลว์ (Abraham Maslow) อธิบายถึงพฤติกรรมของมนุษย์ว่าจะมีความต้องการ
เป็นระดับต่าง ๆ เรียงลำดับจากความต้องการระดับพื้นฐานไปยังระดับสูงสุด 8 ระดับ ดังนี้

\begin{enumerate}
  \item \textbf{Physiological Needs}
  
    คือความต้องการพื้นฐานเพื่ออยู่รอด ได้แก่ ความต้องการอาหาร น้ำ อากาศ 
  ที่อยู่อาศัย ตลอดทั้งมีสภาพแวดล้อมการทำงานที่เหมาะสม

  \item \textbf{Safety Needs}
  
    คือความต้องการสภาพแวดล้อมที่ปลอดจากอันตรายทั้งทางกายและจิตใจ

  \item \textbf{Love and Belonging Needs}
  
    คือความต้องการเป็นเจ้าของ ต้องการความสัมพันธ์ ความรัก มิตรภาพ 
    การได้รับการยอมรับที่จะทำให้มนุษย์รู้สึกถึงการเป็นส่วนหนึ่งของสังคม

  \item \textbf{Self-Esteem Needs}
  
    คือมีความภูมิใจในตนเอง ชื่นชมในความสำเร็จของงานที่ทำ 
    ความรู้สึกมั่นใจในตนเอง และได้รับความเคารพจากผู้อื่น

  \item \textbf{Cognitive Needs}
  
    ความต้องการรู้และเข้าใจตนเอง รวมถึงทัศนคติในการมองสิ่งต่าง ๆ

  \item \textbf{Aesthetic Needs}
  
    คือความต้องเข้าถึงสุนทรียภาพและความสวยงาม 
    มีความสามารถในการมองเห็นความสวยงามของสิ่งต่าง ๆ รอบตัว

  \item \textbf{Self-actualization Needs}
  
    คือความต้องการเพื่อบรรลุเป้าหมายในชีวิต ต้องการความสำเร็จในสิ่งที่ตนเองปรารถนา 
    พัฒนาทักษะความสามารถของตนเองให้ถึงที่สุด

  \item \textbf{Transcendence Needs}
  
    คือความต้องเพื่อการช่วยเหลือผู้อื่น สามารถมีความสุขเมื่อได้ช่วยเหลือผู้อื่นให้ประสบความสำเร็จ
\end{enumerate}

\pagebreak
\subsection{MDA framework\cite{mda04}}
MDA framework หรือ Mechanics-Dynamics-Aesthetics framework 
เป็นเครื่องมือใช้ในการวิเคราะห์เกมที่อธิบายความสัมพันธ์ระหว่างการออกแบบเกมและการพัฒนาเกม 
สามารถแบ่งออกได้เป็น 3 ส่วน ดังนี้ 
\begin{enumerate}
  \item \textbf{Mechanics}
  
    คือกฎและแนวคิดของเกมที่แสดงถึงระบบของเกม เช่น การบังคับตัวละคร เป้าหมายของเกม เป็นต้น

  \item \textbf{Dynamics}
  
    คือการทำให้กฎมีความสนุกมากขึ้น เช่น การเพิ่มความยากหรือความท้าทายของเกม เป็นต้น

  \item \textbf{Aesthetics}
  
    คือความสนุกหรือสุนทรียภาพที่ได้จากการเล่นเกม สามารถจำแนกออกเป็น 8 ประเภท ดังนี้

  \begin{itemize}
    \item \textbf{Sensation} คือเกมที่สนุกจากภาพ เสียง หรือความรู้สึกต่าง ๆ จากการเล่น
    \item \textbf{Fantasy} คือสิ่งที่ทำให้ผู้เล่นเชื่อหรือมีความรู้สึกร่วมไปกับสถานการณ์ภายในเกม
    \item \textbf{Narrative} คือเกมที่มีการเล่าเนื้อเรื่องเพื่อให้ผู้เล่นมีความรู้สึกร่วมไปกับเนื้อเรื่องของเกม
    \item \textbf{Challenge} คือเกมที่มีความท้าทาย สนุกกับการเอาชนะเกม
    \item \textbf{Fellowship} คือเกมที่ทำให้ผู้เล่นได้เข้าสังคม พูดคุย แลกเปลี่ยนกับผู้อื่น
    \item \textbf{Discovery} คือเกมที่ให้ผู้เล่นได้ผจญภัย ออกสำรวจ และค้นพบสิ่งใหม่ในโลกของเกม
    \item \textbf{Expression} คือเกมที่ทำให้ผู้เล่นได้ค้นพบตัวเอง
    \item \textbf{Submission} คือเกมที่สามารถเล่นเป็นงานอดิเรกได้
  \end{itemize}
\end{enumerate}

\subsection{ADDIE Model\cite{addie1}\cite{addie2}}
ADDIE Model คือโมเดลที่ใช้ในการออกแบบกระบวนการเรียนรู้ ประกอบไปด้วย 5 ขั้นตอน ดังนี้
\begin{enumerate}
  \item \textbf{Analyze} คือขั้นตอนการวิเคราะห์และทำความเข้าใจปัญหาการเรียนการสอน 
  วัตถุประสงค์การเรียนรู้ และกลุ่มเป้าหมายที่ต้องการจะสอน
  \item \textbf{Design} คือขั้นตอนการออกแบบจุดประสงค์การเรียนรู้ที่สามารถวัดประเมินผลได้ 
  กลยุทธ์ในการสอน และสื่อต้นแบบ
  \item \textbf{Develop} คือขั้นตอนการพัฒนาการเรียนการสอน สร้างสื่อการเรียนรู้ 
  แบบทดสอบที่จะนำมาวัดประเมินผลการเรียนรู้ตามที่ออกแบบไว้ และทดสอบหาข้อผิดพลาดเพื่อนำมาแก้ไข
  \item \textbf{Implement} คือขั้นตอนการนำสื่อการเรียนรู้ไปสอนกลุ่มเป้าหมายจริงให้เป็นไปตามวัตถุประสงค์ที่ตั้งไว้
  \item \textbf{Evaluate} คือขั้นตอนการประเมินผลการเรียนการสอนว่ารูปแบบการเรียนรู้
  ที่ออกแบบมีจุดใดบ้างที่ควรเพิ่มเติมหรือนำไปปรับปรุงให้รูปแบบการเรียนรู้ดีขึ้นในครั้งหน้า
\end{enumerate}

\pagebreak
\subsection{Bloom Taxonomy\cite{bloom}}
ทฤษฎี Bloom Taxonomy คือทฤษฎีที่กล่าวเกี่ยวกับพฤติกรรมการเรียนรู้
ของมนุษย์ที่เป็นไปตามลำดับขั้น 6 ระดับเรียงจากลำดับล่างสุด ดังนี้

\begin{enumerate}
  \item \textbf{Remember} คือสามารถจดจำ ทบทวนข้อเท็จจริงและข้อมูลพื้นฐานของความรู้ได้
  \item \textbf{Understand} คือเข้าใจและสามารถอธิบายใจความสำคัญของความรู้ได้
  \item \textbf{Apply} คือสามารถนำความรู้ที่มีไปประยุกต์ใช้ได้
  \item \textbf{Analyse} คือสามารถวิเคราะห์ แยกแยะ หาความเชื่อมโยงของส่วนที่คล้ายกัน
  หรือสามารถเปรียบเทียบหาความแตกต่างได้ 
  \item \textbf{Evaluate} คือสามารถประเมินค่าของข้อมูลข้อเท็จจริงได้
  \item \textbf{Create} คือสามารถนำความรู้มาสร้างสิ่งใหม่ให้เกิดขึ้นได้
\end{enumerate}


\subsection{Outcome-based education\cite{obe}}
Outcome-based education (OBE) คือ ทฤษฎีการศึกษาที่มุ่งเน้นการออกแบบ
การระบบการสอนให้ผู้เรียนได้บรรลุผลการเรียนรู้ตามที่ตั้งเป้าหมายไว้ โดยการออกแบบ
ระบบการสอนนี้แบ่งออกเป็น 3 ส่วน ดังนี้
\begin{enumerate}
  \item \textbf{Learning Outcomes} คือวัตถุประสงค์การเรียนรู้ที่ตั้งขึ้นเพื่อเป็นแนวทางในการสอน
  \item \textbf{Assessment Method} คือเกณฑ์การประเมินผลว่าผู้เรียนได้รับความรู้หรือทักษะตาม
  ที่ตั้งไว้ตามวัตถุประสงค์การเรียนรู้มากน้อยเพียงใด
  \item \textbf{Teaching / Learning approaches} คือกลยุทธ์วิธีการสอนที่สร้างขึ้นเพื่อทำให้
  ผู้เรียนสามารถบรรลุผลการเรียนรู้ได้
\end{enumerate}


\subsection{PDCA\cite{pdca10}}
PDCA เป็นวงจรการบริหารจัดการ วางแผนแก้ปัญหาและควบคุมคุณภาพ ทำให้เกิดการพัฒนาอย่างต่อเนื่อง ประกอบไปด้วย 4 ขั้นตอน ดังนี้
\begin{enumerate}
  \item \textbf{Plan} คือขั้นตอนการวางแผนการดำเนินงาน
  \item \textbf{Do} คือขั้นตอนการดำเนินการตามแผนการที่วางไว้
  \item \textbf{Check} คือขั้นตอนการตรวจสอบสิ่งที่ดำเนินการไปแล้วว่ามีสิ่งใดที่ต้องปรับปรุงแก้ไขหรือไม่
  \item \textbf{Act} คือขั้นตอนการดำเนินการแก้ไขปรับปรุง
\end{enumerate}


\pagebreak
\subsection{Reinforcement Learning\cite{rl18}}
เป็นวิธีการเรียนรู้ของ Machine Learning แบบหนึ่ง 
มีหลักการทำงานคือลองผิดลองถูกในการเรียนรู้ และได้รับรางวัลหรือบทลงโทษจากการกระทำนั้น ๆ 
ซึ่งมีองค์ประกอบต่าง ๆ ได้แก่
\begin{enumerate}
  \item \textbf{Agent} คือ AI ที่จะเรียนรู้
  \item \textbf{Action} คือการกระทำของ Agent
  \item \textbf{Environment} คือสภาพแวดล้อมหรือเหตุการณ์จำลองที่ต้องการให้ Agent เรียนรู้
  \item \textbf{State} คือสิ่งที่ Agent รับรู้ได้เพื่อเป็นตัวช่วยในการตัดสินใจเลือก Action
  \item \textbf{Reward} คือผลลัพธ์จากการประเมิน Action ที่ Agent เลือกกระทำ
\end{enumerate}

กระบวนการเรียนรู้คือ Agent พยายามที่จะทำให้ตัวเองได้รางวัลให้ได้มากที่สุดจากการพิจารณา 
State และตัดสินใจเลือก Action ส่งผ่านเข้าไปยัง Environment จากนั้นจะได้รับ State 
ที่เปลี่ยนแปลง และค่า Reward เพื่อเป็นปัจจัยในการตัดสินใจเลือก Action ในครั้งถัดไป ดังรูปที่~\ref{fig:2-2}

\begin{figure}[H]\centering
  \includegraphics[width=10cm]{./imgs/2-2.png}
  \caption{แผนภาพการทำงานของ Reinforcement Learning}\label{fig:2-2}
  \small [ที่มา : \url{https://medium.com/asquarelab/ep-1-reinforcement-learning-เบื้องต้น-acfa9d42394c}]
\end{figure}

\pagebreak
\section{ภาษาคอมพิวเตอร์และเทคโนโลยี}

\subsection{C\#}
เป็นภาษาที่สามารถใช้งานร่วมกับ Unity Engine framework 
เพื่อใช้เขียนโปรแกรมให้ Object สามารถทำงานได้ตามที่ต้องการ 
โดยใช้ Object-Oriented-Programming Pattern เป็นหลัก
\subsection{Unity NetCode}
คือ framework ที่สร้างมาสำหรับ Unity Engine เพื่อใช้พัฒนาระบบ 
Multiplayer ภายในเกม 
\subsection{Unity ML-Agent}
เป็นเครื่องมือของ Unity ที่สามารถนำมาสร้าง Machine Learning 
ได้ด้วยวิธีต่าง ๆ รวมถึง Reinforcement Learning และสามารถสร้าง 
Environment ขึ้นมาเองเพื่อใช้เทรน AI ได้


\section{เครื่องมือในการพัฒนา}

\subsection{Figma}
เป็นเครื่องมือในการออกแบบ UI ที่มีฟีเจอร์การใช้งานหลากหลาย 
ตั้งแต่งานด้าน Graphic Design และ UX/UI Design สามารถใช้ทำ 
Prototype และกราฟิกแบบเวกเตอร์ที่ใช้งานได้บนเว็บเบราว์เซอร์เป็นหลัก 

\subsection{Unity}
เป็น Game Engine ที่รองรับภาษา C\# เพื่อใช้เขียนโปรแกรมพัฒนาเกม  
สามารถทำเกมได้ทั้งรูปแบบ 2D และ 3D มีเครื่องมืออำนวยความสะดวก 
เช่น animator, cinematic เป็นต้น

\subsection{Adobe Photoshop}
เป็นโปรแกรมตัดต่อรูปภาพที่ใช้กันอย่างแพร่หลาย รองรับวาดภาพ Digital Art 
ภายในโปรแกรม และสามารถ export เพื่อใช้งานกับ Unity ได้โดยตรง
 
\subsection{Adobe Illustrator}
เป็นโปรแกรมสำหรับวาดภาพประเภท vector ที่เมื่อขยายภาพแล้วจะยังคงมีความคมชัด
เท่าเดิม สามารถ export ภาพหลายขนาดในครั้งเดียวได้
 
\subsection{Medibang Paint} 
เป็นโปรแกรม Freeware สำหรับวาดภาพ Digital Art มีการใช้งานที่ง่าย 
สามารถเก็บภาพที่วาดไว้บนระบบ Medibang Cloud เพื่อใช้งานแบบ Cross-platform 
ร่วมกับอุปกรณ์เครื่องอื่น ๆ

\subsection{DragonBones} 
เป็นโปรแกรม Open source สำหรับทำ Animation 
สามารถใช้สร้าง Animation 2D จากภาพที่แยกไว้เป็นส่วนต่าง ๆ ได้ง่ายและสะดวก


\pagebreak
\section{ตัวอย่างของผลงานที่เกี่ยวข้อง}
\textbf{ประเภทเกมที่เกี่ยวข้อง}
\begin{itemize}
  \item \textbf{Strategy} คือเกมประเภทวางแผนเพื่อเอาชนะฝ่ายตรงข้าม 
  หรือบรรลุเป้าหมายภายในเกม โดยส่วนใหญ่เป็นแบบผลัดตาเล่น (Turn-Based) 
  ที่ผู้เล่นจะมีเวลาเพื่อวางแผนหรือเล่นเกมในตาเล่นของตนเอง
  \item \textbf{Real-time Strategy} เป็นประเภทเกมที่ผู้เล่นสามารถวางแผน
  และดำเนินตามแผนโต้ตอบกับสถานการณ์ในเกมได้ทันที มีความแตกต่างกับประเภท Strategy 
  ในด้านของเวลา เพราะ Strategy สามารถวางแผนในตาเล่นของตนเองโดยไม่ต้องห่วงการกระทำ
  ของผู้เล่นคนอื่น แต่สำหรับ Real-time Strategy ผู้เล่นต้องวางแผนพร้อมกับหาทางรับมือการโจมตี
  ของผู้เล่นอื่นไปด้วยในเวลาเดียวกัน
  \item \textbf{Tower Defense} คือประเภทย่อยของเกมวางแผนกลยุทธ์ที่มีเป้าหมายคือผู้เล่นต้อง
  ป้องกันดินแดนหรือทรัพย์สินจากศัตรูโดยขัดขวางหรือหยุดศัตรูไม่ให้ไปถึงทางออก 
  มักจะทำได้โดยการวางสิ่งก่อสร้างประเภทต่าง ๆ เพื่อบล็อค ขัดขวาง โจมตี 
  หรือทำลายศัตรูตามเส้นทางการโจมตี
\end{itemize}

\subsection{Kingdom Rush}
Kingdom Rush เป็นเกมที่ผู้เล่นจะต้องปกป้องเส้นทางจากศัตรูที่เดินเข้ามาโดยการสร้าง 
Tower 4 ประเภท ได้แก่ นักเวทย์ นักธนู ค่ายทหาร และปืนใหญ่ โดยเกมนี้มีลักษณะเด่น 
คือ ผู้เล่นสามารถเคลื่อนย้ายตัวละคร Hero เพื่อไปต่อสู้ในตำแหน่งที่ต้องการได้ และ 
ผู้เล่นสามารถใช้สกิลพิเศษ เช่น อุกกาบาต ในการทำลายศัตรูเป็นวงกว้างได้
\begin{figure}[H]\centering
  \includegraphics[width=7cm]{./imgs/2-3.jpeg}
  \caption{เกม Kingdom Rush}\label{fig:2-3}
  \small [ที่มา : \url{https://store.steampowered.com/app/246420/Kingdom_Rush___Tower_Defense}]
\end{figure}

\pagebreak
\subsection{Plant VS Zombies}
Plants vs. Zombies เป็นเกมที่ผู้เล่นจะต้องปกป้องบ้านจาก Zombie 
โดยการวาง Plant ต่างๆ ลงในสนามหญ้าจะที่ถูกแบ่งออกเป็นตารางโดยความน่าสนใจของเกมนี้ 
คือ Zombie ที่เป็นศัตรูสามารถทำลาย Plant ที่เป็น Tower ได้ และการ Plant กับ Zombie 
นั้นจะโจมตีกันภายในเส้นทางเดียวกันเท่านั้น\\
\textbf{ฟีเจอร์ที่นำมาปรับใช้ : } Unit ที่สามารถสร้างพลังงาน โดยพลังงานที่สร้างขึ้นจาก 
Unit นั้นสามารถนำไปใช้ในเกมได้ เช่น ใช้เพิ่มความสามารถตัวละคร  
ใช้เรียก Unit อื่น ๆ เพิ่ม เป็นต้น
\begin{figure}[H]\centering
  \includegraphics[width=7cm]{./imgs/2-4.jpg}
  \caption{เกม Plant VS Zombies}\label{fig:2-4}
  \small [ที่มา : \url{https://store.steampowered.com/app/3590/Plants_vs_Zombies_GOTY_Edit}]
\end{figure}

\subsection{Genshin Impact - Theater Mechanicus event}
Genshin Impact - Theater Mechanicus event เป็นอีเวนท์ในเกมที่ผู้เล่น
จะต้องป้องกันมอนสเตอร์ไม่ให้เข้าประตูมิติ โดยอีเวนท์นี้มีความแตกต่างจากเกมอื่น 
คือ การมีระบบธาตุเข้ามาเกี่ยวข้องซึ่งผู้เล่นสามารถช่วย Tower ได้โดยการโจมตีให้
มอนสเตอร์ติดธาตุ (ผู้เล่นจะไม่สามารถทำความเสียหายได้โดยตรง) 
และการทำปฏิกิริยาธาตุเมื่อมอนสเตอร์ตัวนั้นติดธาตุมากกว่า 2 ธาตุขึ้นไป เช่น 
การแช่แข็งที่เกิดจากการติดธาตุน้ำและน้ำแข็ง การโอเวอร์โหลด (ระเบิด) ที่เกิดจากการติดธาตุไฟและไฟฟ้า เป็นต้น\\
\textbf{ฟีเจอร์ที่นำมาปรับใช้ : } ความสามารถพิเศษตัวละคร เช่น สกิลที่สามารถสร้างขนมโดยใช้พลังงานน้อยลง
\begin{figure}[H]\centering
  \includegraphics[width=10cm]{./imgs/2-5.jpg}
  \caption{เกม Genshin Impact - Theater Mechanicus event}\label{fig:2-5}
  \small [ที่มา : \url{https://www.unpause.asia/2021/02/genshin-impact-how-to-play-theater-mechanicus/2}]
\end{figure}

\pagebreak
\subsection{Orcs must die! 3}
Orcs must die! เป็นเกม 3D มุมมองบุคคลที่ 3 ที่ผู้เล่นจะต้องปกป้อง Rifts จาก Orc โดยเกมนี้มีจุดเด่น คือ 
ผู้เล่นจะต้องบังคับตัวละครในการวางกับดักและช่วยยิง Orc ทำให้ผู้เล่นมีส่วนร่วมในเกมมากขึ้น มีระบบ Multiplayer 
ทำให้สามารถเล่นร่วมกันได้สูงสุด 2 คน\\
\textbf{ฟีเจอร์ที่นำมาปรับใช้ : } ระบบการเล่นหลายคนที่ผู้เล่นสามารถเข้ามาร่วมกันวางแผนต่อสู้ และสนุกไปด้วยกันได้
\begin{figure}[H]\centering
  \includegraphics[width=10cm]{./imgs/2-6.jpg}
  \caption{เกม Orcs must die! 3}\label{fig:2-6}
  \small [ที่มา : \url{https://store.steampowered.com/app/1522820/Orcs_Must_Die_3}]
\end{figure}

\subsection{Clash of Clan}
Clash of Clan เป็นเกมที่ผู้เล่นจะต้องพัฒนา วางแผนป้องกันเมืองของตนเองจากผู้เล่นอื่นที่จะสุ่มมาโจมตี 
โดยวางสิ่งก่อสร้างหรือกับดักต่าง ๆ ไว้รอบ ๆ เมือง และสร้างกองกำลังทหารเพื่อไปโจมตีเมืองของผู้เล่นอื่น
เพื่อปล้นทรัพยากรมาพัฒนาเมืองต่อไป\\
\textbf{ฟีเจอร์ที่นำมาปรับใช้ : } การเพิ่มระดับความสามารถตัวละครภายในเกม เช่น 
เพิ่มความเร็วในการเคลื่อนที่ของตัวละคร
\begin{figure}[H]\centering
  \includegraphics[width=10cm]{./imgs/2-7.jpg}
  \caption{เกม Clash of Clan}\label{fig:2-7}
  \small [ที่มา : \url{https://apptrigger.com/2020/11/25/clash-clans-december-update-release}]
\end{figure}

\pagebreak
\subsection{Random dice}
Random dice เป็นเกมที่ผู้เล่นจะต้องจัดทีมลูกเต๋าไปสุ่มต่อสู้หาผู้อยู่รอดกับผู้เล่นอื่น 
เมื่อเริ่มเกมผู้เล่นจะต้องกดปุ่มสุ่มวางตำแหน่งลูกเต๋าจากที่จัดทีมไว้ และ 
ผสมลูกเต๋าที่ประเภทและแต้มเหมือนกันเพื่ออัพเกรด\\
\textbf{ฟีเจอร์ที่นำมาปรับใช้ : } การอัพเกรด Unit ภายในเกมโดยเมื่ออัพเกรด Unit 
แล้วจะทำให้ความสามารถเฉพาะ Unit นั้น ๆ เพิ่มขึ้น เช่น มีพลังการโจมตีที่มากขึ้น
\begin{figure}[H]\centering
  \includegraphics[height=6cm]{./imgs/2-8.jpeg}
  \caption{เกม Random dice}\label{fig:2-8}
  \small [ที่มา : \url{https://download.heaven32.com/downloads/random-dice-5-6-5-download-for-android-apk-free}]
\end{figure}


\subsection{Clash Royale}
Clash Royale เป็นเกมที่ผู้เล่นจะต้องจัดการ์ด 8 ใบ เพื่อนำไปต่อสู้กับผู้เล่นอื่น 
โดยผู้เล่นจะสามารถใช้การ์ดที่แบ่งออกเป็น 3 ประเภท ได้แก่ กองกำลัง เวทมนตร์ และสิ่งก่อสร้าง 
ซึ่งแต่ละการ์ดจะใช้พลังงานตามจำนวนที่ระบุบนการ์ด และพลังงานจะเพิ่มขึ้นตามเวลา 
ผู้เล่นที่สามารถทำลายป้อมปราการของฝ่ายตรงข้ามได้ก่อนจะเป็นฝ่ายชนะ\\
\textbf{ฟีเจอร์ที่นำมาปรับใช้ : } การนำพลังงานที่ถูกสร้างขึ้นไปใช้ในการสร้าง 
Unit หรือ Tower เพื่อนำมาช่วยต่อสู้ภายในเกม
\begin{figure}[H]\centering
  \includegraphics[height=6cm]{./imgs/2-9.png}
  \caption{เกม Clash Royale}\label{fig:2-9}
  \small [ที่มา : \url{https://apps.apple.com/th/app/clash-royale/id1053012308}]
\end{figure}

\pagebreak
\subsection{Realm of Valor (ROV)}
Realm of Valor เป็นเกมต่อสู้ระหว่างผู้เล่นหลายคนเพื่อทำลายฐานทัพของทีมตรงข้าม 
โดยตามป่าในแผนที่จะมีสิ่งก่อสร้างหรือสิ่งมีชีวิตให้ใช้ประโยชน์เพื่อสร้างความได้เปรียบภายในเกม 
ผู้เล่นต้องร่วมมือและช่วยเหลือกันภายในทีม เพื่อวางแผนและเอาชนะฝ่ายตรงข้าม\\
\textbf{ฟีเจอร์ที่นำมาปรับใช้ : } ตัวละครจะได้รับสถานะพิเศษจากการเอาชนะมอนสเตอร์ป่า 
เพื่อเพิ่มความได้เปรียบภายในเกม เช่น ได้รับพลังโจมตีเพิ่มขึ้นทำให้ทำลาย Unit ของฝ่ายตรงข้ามได้เร็วขึ้น เป็นต้น
\begin{figure}[H]\centering
  \includegraphics[width=8cm]{./imgs/2-10.jpeg}
  \caption{เกม Realm of Valor (ROV)}\label{fig:2-10}
  \small [ที่มา : \url{https://apkfab.com/garena-rov-mobile-moba/com.garena.game.kgth}]
\end{figure}


%%%%%%%%%%%%%%%%%%%%%%%%%%%%%%%%%%%%%%%%%%%%%%%%%%%%%
%%%%%%%%%%%%%%%%%%%%%%%%%%%%%%%%%%%%%%%%%%%%%%%%%%%%%
%%%%%%%%%%%%%%%%%%%%%%%%%%%%%%%%%%%%%%%%%%%%%%%%%%%%%
\chapter{การออกแบบและระเบียบวิธีวิจัย}

ในบทนี้จะกล่าวถึงการออกแบบและวิธีการพัฒนาเกมโดยใช้ ADDIE Model 
ซึ่งจะแบ่งเป็น 5 หัวข้อ ได้แก่ Analysis Design Develop Implement 
และ Evaluate โดยมีเป้าหมายในการออกแบบเพื่อให้ได้เกมที่ผู้เล่นสามารถฝึก
การวางแผนด้วยหลักการ PDCA และสนุกไปกับเกมระหว่างเรียนรู้


\section{Analysis}
ในขั้นตอนนี้จะกล่าวถึงการวิเคราะห์ปัญหาในปัจจุบัน และกลุ่มเป้าหมายของเกม

\subsection{Target Audience Analysis}
กลุ่มผู้เล่นที่มีช่วงอายุตั้งแต่ 15 - 24 ปี อยู่ในช่วงระดับมัธยมศึกษาตอนปลาย
และนักศึกษามหาวิทยาลัย เนื่องจากช่วงอายุดังกล่าวเป็นสัดส่วนที่มากที่สุดจากสัดส่วนของคนที่เล่นเกม 

\subsection{Problem Statement}
การวางแผนและการตัดสินใจเป็นทักษะที่จำเป็น แต่การเรียนรู้หรือทดลอง
จากสถานการณ์จริงสามารถทำได้ยาก เห็นภาพไม่ชัดเจน และวัดผลยากเกินไป

\subsection{Analyze existing teaching methods}
การฝึกทักษะดังกล่าวสามารถทำได้ผ่านการเล่นบอร์ดเกม แต่การเล่นบอร์ดเกม
เพื่อฝึกทักษะ ต้องใช้อุปกรณ์จำนวนมาก และต้องใช้เวลาในการเตรียมการนาน 
ดังนั้นจึงควรมีรูปแบบที่ง่ายต่อการใช้งาน ใช้เวลาในการเตรียมการน้อย 
ไม่ต้องมีการเตรียมอุปกรณ์


\section{Design}
การออกแบบเกมจะเริ่มจากออกแบบ Learning Outcome หลังจากนั้นจะนำทฤษฎีต่าง ๆ 
ที่กล่าวในบทที่ 2 มาประยุกต์ใช้เพื่อออกแบบความสนุกและความท้าทายของเกม 
แล้วออกแบบระบบการเล่นเกมรวมถึงภาพประกอบและตัวละครภายในเกม

\subsection{Learning Outcomes Design}
\begin{itemize}
  \item \textbf{Learning Outcome} ผู้จัดทำได้ออกแบบวัตถุประสงค์การการเรียนรู้
  ของเกมโดยจัดประเภทออกมาเป็น Bloom Taxonomy ได้ดังนี้

  \begin{itemize}
    \item \textbf{LO1 Remember :} สามารถจดจำหลักการวางแผน
    และความหมายแต่ละตัวของ PDCA ได้
    \item \textbf{LO2 Understand :} สามารถอธิบายหลักการ PDCA ได้
    \item \textbf{LO3 Apply :} สามารถวางแผนได้ โดยใช้หลักการ PDCA
    ซึ่งเกณฑ์การประเมินผล
  \end{itemize}

  \item \textbf{Assessment Method} การวัดผลการเรียนรู้จาก 
  Learning Outcome จะแบ่งเป็นระดับคะแนน ดังนี้

  \begin{table}[H]
    \caption{ระดับคะแนนแต่ละ Learning Outcome}\label{tbl:tab-3}
    \centering
    \includegraphics[width=10cm]{./imgs/tbl-3.jpg}
  \end{table}

  \textbf{รายละเอียดการให้คะแนนเพิ่มเติมของ LO3}

  คะแนนของเกณฑ์การประเมิน LO3 นั้นแบ่งออกเป็น 2 ส่วนรวมกันได้ 10 คะแนน ได้แก่

  \begin{enumerate}
    \item \textbf{ช่วง Plan และ Do :} Action score (5 คะแนน) 
    คือคะแนนเมื่อผู้เล่นกระทำ ได้แก่ 

    \begin{itemize}
      \item Unit/Tower Synergy 
      \item การเก็บ objective ย่อยในแผนที่ (เหมืองน้ำตาล มอนสเตอร์ป่า) 
      \item การอัพเกรดต่างๆ
      \item ระดับคะแนน synergy 1-10
    \end{itemize}

    \item \textbf{ช่วง Check และ Act :} Reaction score (5 คะแนน) 
    คือคะแนนที่ผู้เล่นจะได้รับเมื่อทำการตอบโต้แผนฝ่ายตรงข้าม ได้แก่

    \begin{itemize}
      \item การปรับเปลี่ยน Unit/Tower ที่ใช้
      \item อัพเกรดหรือปรับเปลี่ยนสกิลเพื่อแก้ไขสถานการณ์
      \item การทำ action ที่ทำให้เสียเปรียบน้อยลง
    \end{itemize}

  \end{enumerate}

\end{itemize}

\subsection{Game Design}

ผู้จัดทำได้นำทฤษฎีในบทที่ 2 มาประยุกต์ใช้ในการออกแบบเกม ดังนี้

\begin{enumerate}
  \item \textbf{Natural Funativity} 
  
  เพื่อให้ผู้เล่นสนุกไปกับเกม เราสามารถออกแบบความสนุกได้ 3 ด้าน ดังนี้
  
  \begin{itemize}
    \item \textbf{Mental fun :} ผู้เล่นได้สนุกกับการได้คิดและ
    วางแผนสู้กับผู้เล่นคนอื่น มีความรู้สึกท้าทาย
    \item \textbf{Social fun :} เป็นเกมที่ได้เล่นกับผู้เล่นจริง 
    ได้คิดหาวิธีเอาชนะ หรือปรับเปลี่ยนวิธีเพื่อให้ได้เปรียบคู่ต่อสู้
    \item \textbf{Physical fun :} ได้ใช้ทักษะการสังเกตและ
    ใช้มือเพื่อควบคุมตัวละคร ใช้สกิล และอื่นๆ
  \end{itemize}


  \item \textbf{Maslow’s Hierarchy of Needs}
  
  จากทฤษฎีของ Maslow ผู้จัดทำได้ออกแบบความต้องการของผู้เล่นภายในเกมได้เป็น 8 ระดับ ได้แก่

  \begin{itemize}
    \item \textbf{Physiological Needs} 
    
    ในช่วงแรกของเกมผู้เล่นจะต้องหาน้ำตาลมาเพื่อใช้ภายในเกม ทั้งการสร้างหรือการอัพเกรดให้ตนเองเก่งขึ้น

    \item \textbf{Safety Needs} 
    
    เพื่อให้ผู้เล่นสามารถต่อสู้ได้ภายในเกม ผู้เล่นจะต้องสร้าง Unit/Tower เพื่อโจมตีหรือป้องกันจากฝ่ายตรงข้าม รวมถึงการอัพเกรดต่าง ๆ ภายในเกม
    
    \item \textbf{Love and Belonging Needs} 
    
    ผู้เล่นสามารถสะสมวัตถุดิบจากการเล่นเกม แล้วปลดล็อก Unit/Tower ใหม่ได้ และนำ Unit/Tower เหล่านั้นไปใช้เพื่อช่วยต่อสู้
    
    \item \textbf{Self-Esteem Needs} 
    
    เมื่อผู้เล่นสามารถวางแผนและเอาชนะฝ่ายตรงข้ามได้ ผู้เล่นจะรู้สึกภูมิใจในตนเอง มีความมั่นใจในแผนของตนเองมากขึ้น
    
    \item \textbf{Cognitive Needs} 
    
    เมื่อเล่นเกมไปในระยะเวลาหนึ่ง ผู้เล่นจะเริ่มสามารถมองสถานการณ์ในเกมได้รอบคอบและรอบด้านมากขึ้น รวมถึงหาวิธีในการรับมือกับสถานการณ์ในเกมได้
    
    \item \textbf{Aesthetic Needs} 
    
    ผู้เล่นจะได้เพลิดเพลินไปกับภาพในเกมที่อยู่ในรูปแบบของหวานที่น่าดึงดูด ราวกับได้เข้าไปอยู่ในเกม
    
    \item \textbf{Self-actualization Needs} 
    
    ผู้เล่นจะได้ขัดเกลาการวางแผนของตนเองอยู่เสมอ จากการได้เจอผู้เล่นใหม่ ๆ และแผนใหม่ ๆ ทำให้เพิ่มระดับการเล่นของตนเองขึ้นได้
    
    \item \textbf{Transcendence Needs} 
    
    ผู้เล่นสามารถช่วยเหลือผู้อื่นได้จากการส่งต่อความรู้ในการวางแผนและให้คำแนะนำ ทำให้พวกเขาสามารถพัฒนาทักษะให้เก่งขึ้นได้
  \end{itemize}


  \item \textbf{MDA framework}
  
  จากทฤษฎี MDA Framework เราสามารถออกแบบระบบการเล่นได้ ดังนี้

  \begin{itemize}
    \item \textbf{Mechanics} ระบบหลักของเกมคือผู้เล่นจะต้องวางแผน 
    สะสมน้ำตาล สร้าง Unit/Tower เพื่อใช้ต่อสู้รวมถึงอัพเกรดสิ่งต่าง ๆ 
    ให้เก่งขึ้นและเอาชนะฝ่ายตรงข้าม
    \item \textbf{Dynamics} สิ่งที่จะทำให้กฎในหัวข้อ Mechanics 
    มีความสนุกมากขึ้นได้แก่ การได้เจอผู้เล่นที่มีกลยุทธ์หลากหลาย ได้ลองใช้แผนใหม่ ๆ 
    หรือพัฒนาแผนที่มีให้ดียิ่งขึ้น
    \item \textbf{Aesthetics} คือความสนุกภายในเกม สามารถแบ่ง
    ความสนุกเป็นประเภทต่าง ๆ ได้ดังนี้

    \begin{itemize}
      \item \textbf{Sensation}  ผู้เล่นจะได้เพลิดเพลินไปกับภาพเกมที่สวยงาม เสียงประกอบที่เข้ากัน 
      \item \textbf{Fantasy}  ด้วยภาพในธีมของหวานที่น่ารัก ทำให้ผู้เล่นรู้สึกได้มีส่วนร่วมไปกับฉากในเกม
      \item \textbf{Challenge}  ผู้เล่นจะได้สนุกกับการท้าทายแผนใหม่ ๆ หรือเจอกับผู้เล่นที่เก่งกว่า และจะได้รับความภูมิใจเมื่อสามารถเอาชนะเกมได้
      \item \textbf{Fellowship}  ผู้เล่นได้สนุกกับการแข่งขันกับผู้เล่นจริง หรืออาจเล่นแข่งขันกับเพื่อนเพื่อเสริมความสัมพันธ์
      \item \textbf{Discovery}  ผู้เล่นจะได้เพลิดเพลินกับการเดินสำรวจภายในแผนที่เพื่อหาและเก็บพุ่มน้ำตาล เหมืองน้ำตาล หรือแม้แต่ต่อสู้มอนสเตอร์ป่าเพื่อได้รับค่าสถานะพิเศษ
      \item \textbf{Expression}  เมื่อเล่นเกมชนะผู้เล่นจะเกิดความภูมิใจในตัวเอง ได้รับรู้ว่าตนเองสามารถวางแผนต่าง ๆ และทำตามแผนนั้นได้
      \item \textbf{Submission}  เกมสามารถเล่นได้ในระยะเวลาสั้น ๆ ทำให้ผู้เล่นสามารถเล่นเป็นงานอดิเรกได้	
    \end{itemize}
  \end{itemize}
\end{enumerate}

\pagebreak
\subsection{Gameplay and Mechanics}
ในหัวข้อนี้จะกล่าวถึงการออกแบบระบบการเล่นเกม วิธีการเล่นเกม และอธิบายส่วนประกอบต่าง ๆ ของเกม\\

\textbf{Game Goal}
\begin{itemize}
  \item{\textbf{เป้าหมายระยะยาว}}
  \begin{enumerate}
    \item Attacker สามารถทำลายฐานของ Defender ได้สำเร็จ
    \item Defender สามารถป้องกันฐานได้ครบตามเวลาที่กำหนด
    \item ผู้เล่นฝ่ายใดฝ่ายหนึ่งสามารถเก็บน้ำตาลเข้าฐานได้ครบตามกำหนด
  \end{enumerate}
  
  \item{\textbf{เป้าหมายระยะสั้น}}
  \begin{enumerate}
    \item เก็บสะสมน้ำตาลภายในเกมเพื่อนำไปใช้ประโยชน์ในเกม
    \item Upgrade Unit/Tower หรือความสามารถตัวละครให้เก่งขึ้น
    \item รับค่าสถานะพิเศษจากการกำจัด Wild Monster เพื่อสร้างความได้เปรียบ
  \end{enumerate}
\end{itemize}

\textbf{Mechanics} 

\begin{figure}[H]\centering
  \includegraphics[width=9cm]{./imgs/3-1.png}
  \caption{ตัวอย่างภาพรวมของเกม}\label{fig:3-1}
\end{figure}

ภายในเกมจะมีผู้เล่นอยู่ 2 ฝั่ง ดังนี้
\begin{enumerate}
  \item \textbf{Attacker} สามารถเรียก Unit ออกมาเพื่อไปตีฐานของฝั่ง Defender
  \begin{figure}[H]\centering
    \includegraphics[width=1cm]{./imgs/3-2.png}
    \caption{ตัวละคร Attacker}\label{fig:3-2}
  \end{figure}

  \pagebreak
  \item \textbf{Defender} ตั้ง Tower เพื่อป้องกันไม่ให้ Unit ของ Attacker มาตีฐาน
  \begin{figure}[H]\centering
    \includegraphics[width=1cm]{./imgs/3-3.png}
    \caption{ตัวละคร Defender}\label{fig:3-3}
  \end{figure}
\end{enumerate}

\textbf{เงื่อนไขการได้รับน้ำตาล}

\begin{enumerate}
  \item ผู้เล่นแต่ละฝ่ายจะได้รับน้ำตาลต่อวินาที (เข้าฐานโดยตรง)
  \begin{figure}[H]\centering
    \begin{minipage}{.3\textwidth}
      \centering
      \includegraphics[width=3cm]{./imgs/3-4.png}
      \caption{ฐานของ Attacker}\label{fig:3-4}
    \end{minipage}
    \begin{minipage}{.3\textwidth}
      \centering
      \includegraphics[width=3cm]{./imgs/3-5.png}
      \caption{ฐานของ Defender}\label{fig:3-5}
    \end{minipage}
  \end{figure}

  \item เมื่อ \textbf{Unit} ถูกทำลาย จะดรอปน้ำตาล
  ที่ทั้งสองฝ่ายสามารถแย่งชิงกันเก็บเพื่อเอาไปสะสมที่ฐานได้
  \begin{figure}[H]\centering
    \includegraphics[width=5cm]{./imgs/3-6.png}
    \caption{Unit ถูกทำลาย}\label{fig:3-6}
  \end{figure}

  \item พุ่มน้ำตาลและเหมืองน้ำตาลที่สร้างน้ำตาลตามเวลา 
  ที่ผู้เล่นทุกคนสามารถแย่งชิงกันเพื่อนำไปสะสมที่ฐาน
  \begin{figure}[H]\centering
    \begin{minipage}{.3\textwidth}
      \centering
      \includegraphics[height=3cm]{./imgs/3-7.png}
      \caption{พุ่มน้ำตาล}\label{fig:3-7}
    \end{minipage}
    \begin{minipage}{.3\textwidth}
      \centering
      \includegraphics[height=3cm]{./imgs/3-8.png}
      \caption{เหมืองน้ำตาล}\label{fig:3-8}
    \end{minipage}
  \end{figure}
\end{enumerate}

\pagebreak
\textbf{Game Control}
การควบคุมต่าง ๆ ภายในเกมสามารถทำได้โดยใช้คีย์บอร์ดและเมาส์ ดังนี้
\begin{itemize}
  \item \textbf{คลิกซ้าย} : กำหนดจุดที่ตัวละครจะเดินไป
  \item \textbf{คลิกขวา} : interact กับสิ่งต่าง ๆ เช่น สร้าง Unit/Tower หรือเก็บน้ำตาล 
  \item \textbf{ปุ่ม E} : ใช้ skill ที่ 1
  \item \textbf{ปุ่ม F} : ใช้ skill ที่ 2
  \item \textbf{ปุ่ม Esc} : เข้าหน้า game options
  
\end{itemize}

\textbf{Screen Description}

ในหัวข้อนี้จะอธิบายส่วนประกอบของ User Interface (UI) แต่ละหน้าภายในเกม

\begin{itemize}
  \item \textbf{Home} ในหน้าแรกของเกม มีเมนูตามหมายเลข ดังนี้
  \begin{enumerate}
    \item โปรไฟล์ของผู้เล่น
    \item ปุ่มเข้าหน้า Setting
    \item เข้าหน้ากระเป๋าเพื่อดูของที่มีหรือ ปลดล็อค Unit/Tower แบบใหม่
    \item ปุ่มเริ่มเล่นเกม
  \end{enumerate}

  \begin{figure}[H]\centering
    \setlength{\fboxsep}{0cm}
    \fbox{\includegraphics[width=7cm]{./imgs/3-9.png}}
    \caption{หน้า Home}\label{fig:3-9}
  \end{figure}

  \pagebreak
  \item \textbf{Inventory (Bag)} เมื่อผู้เล่นเข้ามาในหน้ากระเป๋า 
  จะเห็นเป็นเมนูตามหมายเลข ดังนี้
  \begin{enumerate}
    \item ปุ่มกลับไปหน้าแรก
    \item แท็บแบ่งไอเทมเป็น 3 ประเภท ได้แก่
    \begin{itemize}
      \item \textbf{Skill} ความสามารถที่ผู้เล่นใช้ในเกมได้
      \item \textbf{Unit/Tower} ตัวละครที่ผู้เล่นสามารถนำไปใช้ในการต่อสู้ในเกม
      \item \textbf{Resource} วัตถุดิบที่ผู้เล่นได้รับจากการเล่นเกม
    \end{itemize}
    \item ไอคอนและชื่อสิ่งที่ผู้เล่นปลดล็อคไว้แล้ว 
    รวมถึงสิ่งที่ยังไม่ปลดล็อค โดยจะเรียงจากสิ่งที่ปลดล็อคแล้วก่อน
    \item เมื่อผู้เล่นนำเมาส์ไปชี้ที่ไอเทมใดไอเทมหนึ่ง 
    จะแสดงออกมาเป็น UI บอกรายละเอียดของไอเทมนั้นด้านซ้าย
  \end{enumerate}

  \begin{figure}[H]\centering
    \setlength{\fboxsep}{0cm}
    \fbox{\includegraphics[width=7cm]{./imgs/3-10.png}}
    \caption{หน้า Inventory}\label{fig:3-10}
  \end{figure}

  หากผู้เล่นนำเมาส์ชี้ไปที่ไอเทมที่ยังไม่ปลดล็อค UI 
  จะแสดงข้อมูลไอเทมรวมถึงการปลดล็อค ดังนี้
  \begin{enumerate}
    \item รายละเอียดของไอเทม
    \item จำนวนวัตถุดิบที่ต้องใช้ รวมถึงปุ่มปลดล็อคไอเทม
  \end{enumerate}

  \begin{figure}[H]\centering
    \includegraphics[width=3cm]{./imgs/3-11.png}
    \caption{UI แสดงข้อมูลปลดล็อก Unit}\label{fig:3-11}
  \end{figure}

  \pagebreak
  \item \textbf{Lobby} เมื่อผู้เล่นกดเริ่มเกม จะเข้าสู่ระบบจับคู่และเข้าสู่หน้าเตรียมตัวเล่นเกม 
  ในหน้านี้ผู้เล่นแต่ละฝ่ายจะต้องเลือก Unit/Tower ที่ต้องการใช้ภายในเกม 
  โดยทั้งสองฝ่ายจะเห็นว่าฝ่ายตรงข้ามเลือก Unit/Tower ใดบ้างไปใช้ในเกม 
  เพื่อเป็นการคาดเดาแผนการและหาทางรับมือกับแผนของอีกฝ่าย โดยจะมีส่วนประกอบตามหมายเลข ดังนี้
  \begin{enumerate}
    \item เวลานับถอยหลังที่เหลือก่อนเริ่มเกม
    \item Unit ที่ฝ่ายตรงข้ามเลือก
    \item Tower ที่ผู้เล่นเลือก
    \item ปุ่มพร้อม ถ้าหากทั้งสองฝ่ายพร้อม เกมจะเริ่มทันที
  \end{enumerate}

  \begin{figure}[H]\centering
    \setlength{\fboxsep}{0cm}
    \fbox{\includegraphics[width=7cm]{./imgs/3-12.png}}
    \caption{หน้า Lobby}\label{fig:3-12}
  \end{figure}
  
  \item \textbf{Gameplay} เมื่อผู้เล่นเข้าเกมมาแล้ว จะเจอกับองค์ประกอบตามหมายเลข ดังนี้
  \begin{enumerate}
    \item ปุ่มเข้าหน้า Option ของเกม
    \item เวลาที่เหลือภายในเกม
    \item เมนูเมื่อกดคลิกที่ฐานของตนเอง โดยเมนูนี้จะปรากฎเมื่อผู้เล่นคลิกเท่านั้น ประกอบไปด้วย
    \begin{itemize}
      \item ปุ่มฝากน้ำตาลเข้าฐาน
      \item ปุ่มถอนน้ำตาลจากฐาน
      \item เปิดร้านค้า
    \end{itemize}
    \item เมื่อผู้เล่นคลิกบนแผนที่ จะมีเมนูการสร้างขึ้นมา 
    โดยจะมี Unit/Tower ให้เลือกสร้างพร้อมระบุจำนวนน้ำตาลที่ใช้
    \item UI แสดงสกิลของผู้เล่น จะแสดงจำนวนคูลดาวน์เมื่อสกิลยังไม่พร้อมใช้งาน
  \end{enumerate}

  \begin{figure}[H]\centering
    \setlength{\fboxsep}{0cm}
    \fbox{\includegraphics[width=7cm]{./imgs/3-13.png}}
    \caption{หน้า Gameplay}\label{fig:3-13}
  \end{figure}
  

  \item \textbf{Shop ใน Gameplay} เมื่อผู้เล่นเปิดหน้าร้านค้าของเกม จะแสดงเมนูตามหมายเลข ดังนี้
  \begin{enumerate}
    \item แท็บแยกประเภทสินค้าได้แก่ Unit/Tower และ Skill
    \item ไอเทมที่สามารถซื้อ/อัพเกรดได้
    \item จำนวนน้ำตาลที่มี
    \item ปุ่มปิดร้านค้า
    
    \begin{figure}[H]\centering
      \setlength{\fboxsep}{0cm}
      \fbox{\includegraphics[width=7cm]{./imgs/3-14.png}}
      \caption{หน้า Shop แท็บ Unit/Tower}\label{fig:3-14}
    \end{figure}

    \item Passive Skill เป็นสกิลที่ไม่ต้องกดใช้
    \item Active Skill เป็นสกิลที่ต้องกดใช้งาน
    \item แสดงสกิลที่เลือกปัจจุบันจำนวน 2 สกิล สามาถกดรูป x 
    ด้านขวาบนของสกิลเพื่อยกเลิกการติดตั้งสกิล

    \begin{figure}[H]\centering
      \setlength{\fboxsep}{0cm}
      \fbox{\includegraphics[width=7cm]{./imgs/3-15.png}}
      \caption{หน้า Shop แท็บ Skill}\label{fig:3-15}
    \end{figure}

  \end{enumerate}
  
  \pagebreak
  \item \textbf{Game Option} เมื่อผู้เล่นกดปุ่ม Options จะแสดงเมนูดามหมายเลข ดังนี้
  \begin{enumerate}
    \item ปุ่มปิดหน้าต่าง Option
    \item การตั้งค่าในเกม ได้แก่ เสียงพื้นหลัง และเสียงเอฟเฟคของเกม
    \item ปุ่มยอมแพ้
  \end{enumerate}

  \begin{figure}[H]\centering
    \setlength{\fboxsep}{0cm}
    \fbox{\includegraphics[width=7cm]{./imgs/3-16.png}}
    \caption{หน้า Game Option}\label{fig:3-16}
  \end{figure}
  
  \item \textbf{Game Result}
  เมื่อจบเกมจะแสดง UI ผลการแพ้ชนะของเกมโดยมีส่วนประกอบตามหมายเลข ได้แก่
  \begin{enumerate}
    \item ผลลัพธ์แพ้ชนะของเกม
    \item วัตถุดิบที่ได้รับเมื่อจบเกม เพื่อใช้ในการปลดล็อคตัวละครใหม่ในหน้าแรก
    \item ปุ่ม OK และกลับไปหน้าแรก
  \end{enumerate}
  
  \begin{figure}[H]\centering
    \setlength{\fboxsep}{0cm}
    \fbox{\includegraphics[width=7cm]{./imgs/3-17.png}}
    \caption{หน้า Game Result}\label{fig:3-17}
  \end{figure}
\end{itemize}

\pagebreak
\subsection{Game Elements}
ในหัวข้อนี้จะอธิบายส่วนประกอบต่าง ๆ ภายในเกมและแผนที่เกม
\begin{itemize}
  
  \item \textbf{Player} ผู้เล่นสามารถเดินได้ตามอิสระภายในแผนที่เกม โดยมีกระเป๋า
  สำหรับเก็บน้ำตาล มีสกิลที่ซื้อหรืออัพเกรดได้จากในร้านค้า
  ผู้เล่นจะไม่ถูกโจมตีจาก Tower หรือ Unit
  
    \begin{figure}[H]\centering
      \begin{minipage}{.3\textwidth}
        \centering
        \includegraphics[height=3cm]{./imgs/3-18.png}
        \caption{ตัวละคร Attacker}\label{fig:3-18}
      \end{minipage}
      \begin{minipage}{.3\textwidth}
        \centering
        \includegraphics[height=3cm]{./imgs/3-19.png}
        \caption{ตัวละคร Defender}\label{fig:3-19}
      \end{minipage}
    \end{figure}
  
  \item \textbf{Base} สถานที่ที่ผู้เล่นสามารถนำน้ำตาลมาเก็บสะสมหรือนำออก 
  ซื้อสกิลตัวละครและ Upgrade Unit หรือ Tower ได้
  
    \begin{figure}[H]\centering
      \begin{minipage}{.3\textwidth}
        \centering
        \includegraphics[height=3cm]{./imgs/3-20.png}
        \caption{ฐานของ Attacker}\label{fig:3-20}
      \end{minipage}
      \begin{minipage}{.3\textwidth}
        \centering
        \includegraphics[height=3cm]{./imgs/3-21.png}
        \caption{ฐานของ Defender}\label{fig:3-21}
      \end{minipage}
    \end{figure}
  
  \item \textbf{Unit} สิ่งที่ Attacker สร้างขึ้นมาโดยจะเดินตามถนนในแผนที่เพื่อไป
  ทำลายฐานของ Defender มีความสามารถต่างกันตามประเภท Unit
    
    \begin{figure}[H]\centering
      \includegraphics[width=3cm]{./imgs/3-22.png}
      \caption{Unit}\label{fig:3-22}
    \end{figure}
    
  \pagebreak  
  \item \textbf{Tower} สิ่งที่ Defender สร้างขึ้นมาเพื่อป้องกันการโจมตีจาก Unit 				
  มีความสามารถต่างกันตามประเภทของ Tower
    
    \begin{figure}[H]\centering
      \includegraphics[width=2cm]{./imgs/3-23.png}
      \caption{Tower}\label{fig:3-23}
    \end{figure}

  \item \textbf{Sugar Bush} พุ่มน้ำตาลที่สร้างน้ำตาลได้ตามเวลาที่กำหนด จะอยู่ในแผนที่เกม
      
    \begin{figure}[H]\centering
      \includegraphics[width=3cm]{./imgs/3-24.png}
      \caption{Sugar Bush}\label{fig:3-24}
    \end{figure}

  \item \textbf{Sugar Mine} เหมืองที่สร้างน้ำตาลได้ตามเวลาที่กำหนด จะอยู่ในแผนที่เกม 
  โดยจำนวนน้ำตาลและเวลาคูลดาวน์หลังจากเก็บไปแล้วมากกว่าพุ่มน้ำตาล
      
    \begin{figure}[H]\centering
      \includegraphics[width=3cm]{./imgs/3-25.png}
      \caption{Sugar Mine}\label{fig:3-25}
    \end{figure}
  
  \item \textbf{Wild Monster} สิ่งมีชีวิตในแผนที่ ที่ Player สามารถต่อสู้แล้วจะได้รับ Buff เพื่อ
  ความได้เปรียบ

    \begin{figure}[H]\centering
      \includegraphics[width=3cm]{./imgs/3-26.png}
      \caption{Wild Monster}\label{fig:3-26}
    \end{figure}
\end{itemize}
  
\subsection{Levels}
ในหัวข้อนี้จะกล่าวถึงการออกแบบ Level Design ซึ่งจะแบ่งเป็น 2 ส่วน 
ได้แก่ Game Difficulty Progression และ Level Design
\begin{enumerate}
  \item \textbf{Game Difficulty Progression}
  ความยากของเกมนั้นจะขึ้นอยู่ระดับของผู้เล่นที่เล่นด้วยกัน แต่ในช่วงแรกจะมีการออกแบบด่านที่เป็น 
  Tutorial ไว้สอนผู้เล่นเพื่อทำความรู้จักและเรียนรู้วิธีเล่นเกมเบื้องต้น 
  และมีด่านสำหรับเล่นแข่งขันจริงอีก 1 ด่าน

  \item \textbf{Level Design}
  ด่านในเกมจะแบ่งออกเป็น 2 ประเภท ได้แก่ ด่านสำหรับสอน และด่านสำหรับเล่นจริง
  \begin{enumerate}
    \item ด่านสำหรับสอนผู้เล่นจะแบ่งออกเป็น 4 ด่านย่อย ได้แก่
    \begin{enumerate}
      \item สอนระบบพื้นฐานของเกม การเดิน การหาน้ำตาล และวิธีการจบเกม
      \item สอนเรื่องการอัพเกรดต่าง ๆ การใช้ร้านค้า เพื่อให้ผู้เล่นแข็งแกร่งขึ้น
      \item สอนเกี่ยวกับสิ่งที่อยู่ในแผนที่ ได้แก่ พุ่มน้ำตาล เหมืองน้ำตาล มอนสเตอร์ป่า รวมถึง Fog of War
      \item ให้ผู้เล่นทดลองเล่นเกมจากพื้นฐานที่ได้เรียนรู้จากด่าน 1 2 และ 3
    \end{enumerate}

    \item ด่านที่ใช้เล่นจริง เราได้แบ่งฐานของผู้เล่นไว้ตรงข้ามกัน 
    โดยมีทางเดินสำหรับ Unit 2 เส้นทางเพื่อให้สามารถวางแผนได้ซับซ้อนมากขึ้น 
    มีพุ่มน้ำตาลกระจายอยู่ทั่วแผนที่เพื่อให้สามารถหาน้ำตาลได้ไม่ยากจนเกินไป 
    มีเมืองน้ำตาลฝั่งละที่เพื่อความสมดุล แต่ผู้เล่นสามารถขโมยน้ำตาลจากเหมืองของอีกฝั่งได้เสมอ 
    และสุดท้ายนี้ มีมอนสเตอร์ป่าอยู่ที่กลางแผนที่ซึ่งอยู่ภายใต้หมอกหนา 
    ผู้เล่นจะต้องเดินเข้าไปสำรวจใกล้ ๆ จึงจะเจอ 
    และรอบตัวของมอนสเตอร์ป่าจะมีถนนเพื่อให้สามารถวาง Unit/Tower ได้ ดังรูปที่~\ref{fig:3-27} และ รูปที่~\ref{fig:3-28}

    \begin{figure}[H]\centering
      \begin{minipage}{.45\textwidth}
        \centering
        \includegraphics[height=3cm]{./imgs/3-27.png}
        \caption{ตัวอย่างด่านที่ผู้เล่นเห็น}\label{fig:3-27}
      \end{minipage}
      \begin{minipage}{.45\textwidth}
        \centering
        \includegraphics[height=3cm]{./imgs/3-28.png}
        \caption{การจัดวางด่านเบื้องหลัง โดยไม่มี Fog of War}\label{fig:3-28}
      \end{minipage}
    \end{figure}
  \end{enumerate}
\end{enumerate}

\subsection{System Techniques}
เกมที่ผู้จัดทำออกแบบจะใช้ Unity Engine ร่วมกับภาษา C\# 
ในการพัฒนา โดยมีการต่อฐานข้อมูลเพื่อเก็บข้อมูลของผู้เล่น 
และใช้ Unity NetCode เพื่อพัฒนาระบบ Multiplayer 
โดยเกมจะถูก build ลงบนระบบปฏิบัติการ Windows และ MacOS\\

\textbf{Tools and Plug-ins}
\begin{itemize}
  \item Adobe Photoshop
  \item Adobe Illustrator
  \item Medibang Paint
  \item Unity ML-Agent
  \item Unity NetCode
\end{itemize} 

\textbf{Equipment}
\begin{itemize}
  \item เมาส์ปากกายี่ห้อ Wacom Intuos Pen Small
\end{itemize}

\subsection{Diagrams}
การออกแบบระบบของเกม สามารถเขียนออกมาเป็น Diagram ต่าง ๆ ได้ ดังนี้\\

\textbf{System Architecture}

Unity Engine จะเป็นตัวกลางในการสื่อสารกับผู้เล่น 
โดยรับ Input และส่งออกภาพให้กับ Player ในขณะที่เชื่อมต่อกับ 
Unity NetCode ซึ่งเป็นระบบ Multiplayer ของเกม และ DB (Database) 
เพื่อเก็บข้อมูลของผู้เล่น

\begin{figure}[H]\centering
  \includegraphics[width=12cm]{./imgs/arch.jpg}
  \caption{Architecture Diagram}\label{fig:arch}
\end{figure}

\textbf{Component Diagram}

ในระบบของตัวเกมนั้นจะประกอบไปด้วยส่วนประกอบ 6 ส่วน ดังนี้

\begin{enumerate}
  \item \textbf{Map} คือสิ่งที่จะแสดงอยู่ภายในแผนที่เกม ได้แก่
  \begin{enumerate}
    \item \textbf{Player Base} คือฐานของผู้เล่น
    \item \textbf{Wild Monster} คือมอนสเตอร์ที่อยู่ในแผนที่เมื่อที่
    \item \textbf{Interactable} คือสิ่งที่ผู้เล่นต้องกดเพื่อเก็บสะสมน้ำตาลได้แก่ พุ่มน้ำตาล และ เหมืองน้ำตาล
  \end{enumerate}

  \item \textbf{Multiplayer System} คือระบบที่ใช้เชื่อมต่อข้อมูลต่าง ๆ ในเกม
  ของผู้เล่น 2 คนให้ตรงกัน

  \item \textbf{Score System} คือระบบการนับคะแนนภายในเกมประกอบไปด้วยคะแนน 2 ส่วน ได้แก่
  \begin{enumerate}
    \item \textbf{Action score} คือคะแนนที่ผู้เล่นได้รับเมื่อทำ Action ต่าง ๆ ในเกม
    \item \textbf{Reaction score} คือคะแนนที่ผู้เล่นได้รับเมื่อปรับเปลี่ยนแผนการและแก้ไข
    สถานการณ์ภายในเกมได้
  \end{enumerate}

  \item \textbf{Dynamic Objects} คือส่วนประกอบภายในเกมที่เคลื่อนที่ได้หรือมีข้อมูลที่
  เปลี่ยนแปลงตลอดภายในเกม ได้แก่
  \begin{enumerate}
    \item \textbf{Player} คือข้อมูลต่าง ๆ ของผู้เล่น
    \item \textbf{Unit/Tower} คือตัวละครที่ผู้เล่นจะใช้ในการโจมตีเพื่อบุกฐานหรือป้องกันฐาน
    \item \textbf{Timer} คือนาฬิกาจับเวลาที่จะคอยนับเวลาถอยหลังภายในเกม
  \end{enumerate}
  
  \pagebreak
  \item \textbf{Shop} คือร้านค้าที่ผู้เล่นสามารถนำเงินในเกม(น้ำตาล) 
  เพื่อซื้อของจากร้านค้าได้ ประกอบด้วย
  \begin{enumerate}
    \item \textbf{Skill Shop} คือหน้าร้านค้าที่ผ้เล่นสามารถซื้อและสวมใส่ Skill ของผู้เล่น
    \item \textbf{Unit/Tower Shop} คือร้านค้าที่ผู้เล่นจะใช้น้ำตาล อัพเกรดเลเวล Unit/Tower 
  \end{enumerate}

  \item \textbf{Database} เป็นฐานข้อมูลภายนอกที่เก็บข้อมูลของผู้เล่น
\end{enumerate}

\begin{figure}[H]\centering
  \includegraphics[width=14cm]{./imgs/component.png}
  \caption{Component Diagram}\label{fig:component}
\end{figure}

\pagebreak
\textbf{Use Case Diagram}

ภายในเกมหลังจากผู้เล่นเข้าสู่ระบบ จะมีสิ่งที่ผู้เล่นเลือกทำได้ 
คือการเปิดเมนูกระเป๋าของผู้เล่น ซึ่งจะสามารถดูไอเทมที่มีหรือปลดล็อคไอเทมใหม่ ๆ ได้ 
และเมนูเริ่มเกมที่จะเข้าสู่การเล่นเกมแล้วให้ผู้เล่นได้เลือก action ต่าง ๆ ในเกม 
และเมื่อเกมสิ้นสุด ผู้เล่นจะได้รับ Reward สำหรับใช้ปลดล็อคไอเทมในกระเป๋า

\begin{figure}[H]\centering
  \includegraphics[width=14cm]{./imgs/usecase.png}
  \caption{Use Case Diagram}\label{fig:usecase}
\end{figure}

\textbf{Database} 

ผู้จัดทำออกแบบให้ฐานข้อมูลเชื่อมต่อจากภายนอก เพื่อเก็บข้อมูลของผู้เล่น โดยเก็บในฐานข้อมูล 1 
ตารางชื่อว่า User ดังตาราง~\ref{tbl:database}

\begin{table}[H]
  \caption{User table}\label{tbl:database}
  \begin{tabular}{|l|c|l|l|c|}
  \hline
  \rowcolor[HTML]{C0C0C0} 
  \multicolumn{1}{|c|}{\cellcolor[HTML]{C0C0C0}\textbf{Attribute}} & \textbf{Key} & \multicolumn{1}{c|}{\cellcolor[HTML]{C0C0C0}\textbf{Description}} & \multicolumn{1}{c|}{\cellcolor[HTML]{C0C0C0}\textbf{\begin{tabular}[c]{@{}c@{}}Data Type\\ \& Length\end{tabular}}} & \textbf{\begin{tabular}[c]{@{}c@{}}Nulls\\ (Y/N)\end{tabular}} \\ \hline
  \rowcolor[HTML]{FFFFFF} 
  Id                                                               & PK           & เลขประจำตัวของผู้เล่น                                             & char(10)                                                                                                            & N                                                              \\ \hline
  \rowcolor[HTML]{FFFFFF} 
  username                                                         &              & ชื่อผู้ใช้ของผู้เล่น                                              & varchar(20)                                                                                                         & N                                                              \\ \hline
  \rowcolor[HTML]{FFFFFF} 
  password                                                         &              & รหัสผ่านของผู้เล่น                                                & varchar(30)                                                                                                         & N                                                              \\ \hline
  \rowcolor[HTML]{FFFFFF} 
  profileImg                                                       &              & หมายเลขภาพโปรไฟล์ที่ผู้เล่นเลือกใช้ในเกม                          & int                                                                                                                 & N                                                              \\ \hline
  \rowcolor[HTML]{FFFFFF} 
  finishedTutorial                                                 &              & ผู้เล่นได้ผ่านโหมดการฝึกเล่นแล้วหรือไม่                           & boolean                                                                                                             & N                                                              \\ \hline
  \rowcolor[HTML]{FFFFFF} 
  rank                                                             &              & อันดับคะแนนของผู้เล่น                                             & int                                                                                                                 & N                                                              \\ \hline
  \rowcolor[HTML]{FFFFFF} 
  win                                                              &              & จำนวนเกมที่ผู้เล่นชนะ                                             & int                                                                                                                 & N                                                              \\ \hline
  \rowcolor[HTML]{FFFFFF} 
  lose                                                             &              & จำนวนเกมที่ผู้เล่นแพ้                                             & int                                                                                                                 & N                                                              \\ \hline
  \rowcolor[HTML]{FFFFFF} 
  money                                                            &              & จำนวนส่วนประกอบซึ่งเป็นค่าเงินใช้ปลดล็อคไอเทม                     & int                                                                                                                 & N                                                              \\ \hline
  \rowcolor[HTML]{FFFFFF} 
  unlockedCharJson                                                 &              & ไอเทมหรือตัวละครที่ผู้เล่นปลดล็อค โดยเก็บเป็นรูปแบบ JSON          & varchar                                                                                                             & N                                                              \\ \hline
  \end{tabular}
\end{table}

\pagebreak
\textbf{Class Diagram}

แสดงความสัมพันธ์ของส่วนประกอบต่าง ๆ ที่มีความเกี่ยวข้องกับการเขียนโปรแกรม 
และแสดง property รวมถึง method ของแต่ละ class ซึ่งแบ่งออกมาได้เป็น 4 class ได้แก่

\begin{enumerate}
  \item \textbf{Skill} เก็บระยะเวลา cooldown สถานะของสกิล และดาเมจของ Skill 
  โดยมี method Activate สำหรับเรียกใช้ Skill
  \item \textbf{Player} เก็บค่าพิกัดของผู้เล่น Unit ที่ครอบครอง จำนวนน้ำตาล และ Skill 
  ที่สวมใส่สูงสุด 2 Skill สามารถเดิน สร้าง Unit และปลดล็อค Unit ได้
  \item \textbf{Unit} เป็นตัวละครที่ผู้เล่นครอบครองเพื่อใช้ต่อสู้ จะเก็บชื่อเจ้าของ Unit 
  เป้าหมายที่จะต่อสู้ด้วย จะนวนเลือด และ Skill สูงสุด 2 Skill โดยมี method FindTarget และ MoveToDestination เพื่อหาและเดินเข้าไปต่อสู้ มี method Upgrade เพื่อเพิ่มระดับความสามารถ และ Destroy เพื่อทำลาย Unit หลังแพ้การต่อสู้
  \item \textbf{Interactable} เป็น class ของ object ที่ผู้เล่นสามารถ Interact 
  ด้วยได้โดยจะเก็บค่า cooldown และค่าสถานะของ object และมี method Interact ที่ถูกเรียกเมื่อผู้เล่นคลิก และ DropItem เพื่อให้รางวัลผู้เล่น
\end{enumerate}

\begin{figure}[H]\centering
  \includegraphics[width=15cm]{./imgs/class.png}
  \caption{Class Diagram}\label{fig:class}
\end{figure}

\pagebreak
\textbf{Activity Diagram}

เมื่อผู้เล่นเข้าเกมมาในหน้าแรกผู้เล่นจะต้องกรอก username password เพื่อ login 
เข้าสู่ระบบ หลังจากที่ผู้เล่นเข้าสู่ระบบแล้วระบบจะทำการตรวจสอบกับฐานข้อมูลว่าข้อมูลที่กรอกมาถูกต้องหรือไม่ 
เมื่อกรอกข้อมูลถูกต้องแล้วเกมจะแสดงข้อความว่าผู้เล่น login เข้าสู่เกมได้สำเร็จ เมื่อผู้เล่นกดปุ่มเริ่มเกม 
ระบบเกมจะทำการจับคู่ผู้เล่นเพื่อไปต่อสู้กันระหว่างนั้น ผู้เล่นจะต้องรออยู่ในหน้า Waiting Room 
เมื่อระบบทำการจับคู่ผู้เล่นเสร็จแล้ว ผู้เล่นแต่ละฝ่ายจะผลัดกันเลือก Unit/Tower เมื่อผู้เล่นทั้งสองฝ่ายเลือก 
Unit/Tower เสร็จแล้ว ผู้เล่นก็จะเริ่มเล่นเกมโดยในระหว่างที่เล่นเกมนั้นระบบเกมก็จะทำจับเวลาและ
คอยเก็บคะแนนต่าง ๆ ของผู้เล่น สุดท้ายเมื่อจบเกมระบบก็จะแสดงหน้าผลลัพธ์ของเกมพร้อมทั้งบันทึกข้อมูลของผู้เล่น
ไปเก็บไว้ที่ Database และจะมีปุ่มยืนยันให้ผู้เล่นกดเพื่อกลับไปหน้าแรก

\begin{figure}[H]\centering
  \includegraphics[width=14cm]{./imgs/activity.png}
  \caption{Activity Diagram}\label{fig:activity}
\end{figure}

\section{Development}

ในขั้นตอนการพัฒนา ผู้จัดทำจะทดลอง Proof of Concept (PoC) ก่อนว่าสิ่งที่ออกแบบไว้สามารถทำได้จริง 
หรือถ้าหากเจอปัญหา ก็จะพยายามแก้ไขหรือหาเครื่องมืออื่นที่ใกล้เคียงกันมาทดแทน จากนั้นจะเริ่มทำ Prototype 
เพื่อแสดงภาพรวมเกมเบื้องต้นสำหรับนำไปทดสอบระบบที่มีรายละเอียดมากขึ้น 

เมื่อ Prototype ถูกทดสอบว่าสามารถนำไปใช้ได้ ผู้จัดทำจะเริ่มพัฒนาส่วนต่าง ๆ เพิ่มเติมให้ละเอียดมากขึ้น เช่น 
อนิเมชันตัวละคร ภาพประกอบฉาก ระบบเสียง หรือระบบอื่น ๆ เพื่อเพิ่มความสะดวกให้กับผู้ใช้งาน

ในระหว่างที่พัฒนาเกม ผู้จัดทำจะเริ่มพัฒนา Machine Learning เพื่อนำมาแข่งขันกับผู้เล่นในเกม โดยใช้ 
Unity ML Agent ซึ่งใช้ Reinforcement Learning ในการเรียนรู้อยู่เบื้องหลัง

เครื่องมือหลักที่ใช้ในการพัฒนาคือ Unity Engine ใช้ร่วมกันกับภาษา C\# เพื่อเขียนโปรแกรมระบบเกม 
และใช้ Unity NetCode เพื่อพัฒนาระบบ Multiplayer และ build ลงบนระบบปฏิบัติการ Windows และ MacOS 

\section{Implement}

หลังจากพัฒนาเกมที่เป็นตัวต้นแบบเสร็จ ผู้จัดทำจะทดสอบและทดลองเล่นเพื่อหาจุดผิดพลาดและแก้ไข 
ก่อนนำไปให้กลุ่มเป้าหมายทดลองใช้งานจริงและเก็บ Feedback เพื่อนำมาปรับปรุงต่อไป


\section{Evaluation}

หลังจากการทดสอบและทดลองเล่นจากกลุ่มเป้าหมาย ทางผู้จัดทำจะประเมินผลจากความคิดเห็น
และความพึงพอใจจากกลุ่มเป้าหมายผ่านแบบสำรวจที่ทำขึ้นโดยจะเก็บข้อมูลเบื้องต้นได้แก่ เพศ อายุ 
ระดับชั้นการศึกษาและคำถามประเมินความคิดเห็นของเกมจำนวน 10 คำถามรวมถึงข้อคิดเห็น คำแนะนำจากการทดลองเล่น

%%%%%%%%%%%%%%%%%%%%%%%%%%%%%%%%%%%%%%%%%%%%%%%%%%%%%%%%%%%%%%
%%%%%%%%%%%%%%%%%%%% Experiments %%%%%%%%%%%%%%%%%%%%%%%%%%%%%
%%%%%%%%%%%%%%%%%%%%%%%%%%%%%%%%%%%%%%%%%%%%%%%%%%%%%%%%%%%%%%%
\chapter{ผลการวิจัยและอภิปรายผล}

% //TODO : บรรยายด้วย 

\section{ผลการดำเนินงานในภาคการศึกษาที่ 1} 
\subsection{ปัญหาที่พบ}
  \begin{table}[H]
    \caption{ปัญหาที่พบในภาคการศึกษาที่ 1}\label{tbl:tab-4}
    \begin{tabular}{|c|l|l|}
    \hline
    \textbf{No} & \multicolumn{1}{c|}{\textbf{ปัญหาที่พบ}}                                                                                                                                                 & \multicolumn{1}{c|}{\textbf{วิธีการแก้ไข}}                                                                                                                                                                                         \\ \hline
    1           & \begin{tabular}[c]{@{}l@{}}ไม่สามารถสร้างห้องสำหรับผู้เล่นหลายคนได้ \\ เนื่องจาก Unity NetCode API ไม่รองรับ\\ การสร้าง socket บน Web Browser\end{tabular}                               & \begin{tabular}[c]{@{}l@{}}เปลี่ยน target platform จากเกมที่สามารถเล่นผ่าน Web Browser \\ ให้กลายเป็นเกมสำหรับเล่นบน Windows/MacOS\end{tabular}                                                                                    \\ \hline
    2           & \begin{tabular}[c]{@{}l@{}}NetCode API สำหรับทำระบบ Multiplayer เป็นระบบใหม่ \\ ยังไม่มีข้อมูลหรือตัวอย่างการใช้งานที่มากพอ \\ ทำให้ยากต่อการศึกษาข้อมูลหรือหาตัวอย่างปัญหา\end{tabular} & \begin{tabular}[c]{@{}l@{}}ศึกษาหรือดูจาก API แบบเก่า และนำมาปรับ\\ หรือประยุกต์วิธีการให้เข้ากับ API แบบใหม่\end{tabular}                                                                                                         \\ \hline
    3           & UI ภายในเกมแสดงผลผิดพลาดจากความต่างของขนาดหน้าจอ                                                                                                                                         & \begin{tabular}[c]{@{}l@{}}ปรับ Canvas ให้เป็น Scale with Screen Size \\ และตั้งค่า Anchor ของ UI แต่ละหน้าใหม่ \\ ซึ่ง Anchor จะเป็นตัวควบคุมระยะห่างระหว่าง UI element แต่ละมุม \\ ให้มีระยะห่างเท่าเดิมในทุกขนาดจอ\end{tabular} \\ \hline
    \end{tabular}
  \end{table}
\subsection{คำแนะนำจากอาจารย์}
    \begin{table}[H]
      \caption{คำแนะนำจากอาจารย์}\label{tbl:tab-5}
      \begin{tabular}{|c|l|l|}
      \hline
      \textbf{No} & \multicolumn{1}{c|}{\textbf{ปัญหาที่พบ}}                                                                         & \multicolumn{1}{c|}{\textbf{วิธีการแก้ไข}}                                                                                                                                                 \\ \hline
      1           & \begin{tabular}[c]{@{}l@{}}การออกแบบ Learning Outcome 1 (Remember) \\ จะทำให้ผู้เล่นท่องจำมากเกินไป\end{tabular} & \begin{tabular}[c]{@{}l@{}}ในการออกแบบการทดสอบ Learning Outcome 1 \\ จะไม่ได้เป็นการบังคับผู้เล่นให้ทำและจะเป็นการทำ quiz \\ ตอบคำถามและเมื่อตอบเสร็จแล้วจะมีรางวัลให้ผู้เล่น\end{tabular} \\ \hline
      2           & เกมยังไม่มีการประเมินผลให้เห็นการพัฒนาของผู้เล่น                                                                 & \begin{tabular}[c]{@{}l@{}}มี Score System เพื่อวัดคะแนนการเล่นแต่ละครั้งของผู้เล่น \\ แล้วเก็บเป็นสถิติ และจัดเป็น Ranking ให้ผู้เล่นแข่งขัน\\ เพื่อ Rank ที่สูงขึ้น\end{tabular}         \\ \hline
      \end{tabular}
    \end{table}

\section{ผลการดำเนินงานในภาคการศึกษาที่ 2} 
  \begin{table}[H]
    \caption{ปัญหาที่พบในภาคการศึกษาที่ 2}\label{tbl:tab-6}
    \begin{tabular}{|c|l|l|}
    \hline
    \textbf{No} & \multicolumn{1}{c|}{\textbf{ปัญหาที่พบ}}                                                                                                                                                                      & \multicolumn{1}{c|}{\textbf{วิธีการแก้ไข}}                                                                                                                                                                              \\ \hline
    1           & \begin{tabular}[c]{@{}l@{}}การเพิ่ม Shader ตัวละครหรือรูปภาพในเกม เช่น \\ การใส่สีเส้นขอบ ไม่รองรับรูปภาพที่แยกส่วนไว้ก่อนล่วงหน้า \\ ซึ่งจะถูกแยกไว้สำหรับ Animation ภายในเกม\\ //ภาพ//\end{tabular}         & \begin{tabular}[c]{@{}l@{}}เปลี่ยนจากการทำ Animation แบบจัดกระดูก\\ เป็น Animation frame by frame (Sprite Sheet) \\ โดยใช้โปรแกรม DragonBones\\ //ภาพ//\end{tabular}                                                    \\ \hline
    2           & \begin{tabular}[c]{@{}l@{}}การ Export Animation Sprite Sheet \\ จาก DragonBones แบบหลาย frame ในหนึ่งไฟล์\\ ไม่สามารถเรียงลำดับ frame ได้ถูกต้อง\end{tabular}                                                 & \begin{tabular}[c]{@{}l@{}}เปลี่ยนการ Export เป็นแบบ 1 frame ต่อ 1 ไฟล์ภาพ\\  และจัดเรียงเป็น folder พร้อมตัวเลขกำกับในแต่ละ frame \\ แต่ข้อเสียของวิธีนี้คือจำนวนไฟล์ภาพจะถูก export \\ ออกมาเป็นจำนวนมาก\end{tabular} \\ \hline
    3           & \begin{tabular}[c]{@{}l@{}}ไม่มีความชำนาญด้านการเลือกสี \\ การออกแบบสัญลักษณ์ และการทำ Animation\end{tabular}                                                                                                 & \begin{tabular}[c]{@{}l@{}}ค้นหาภาพ reference มาอ้างอิงในการออกแบบ \\ และใช้ palette สีเพื่อช่วยในการออกแบบ\end{tabular}                                                                                                \\ \hline
    4           & \begin{tabular}[c]{@{}l@{}}การ synchronize ข้อมูลระหว่างผู้เล่น เช่นการเก็บไอเทม \\ ผู้เล่นที่เป็น Client ไม่สามารถเก็บไอเทมได้เนื่องจาก\\ สิทธิ์ในการปรับเปลี่ยนแก้ไขข้อมูลของทุกคนอยู่ที่ Host\end{tabular} & \begin{tabular}[c]{@{}l@{}}จำเป็นต้องให้ Client ส่ง request ไปยัง Host \\ เพื่อขอเก็บไอเทม และต้องตรวจสอบเงื่อนไข\\ ให้แน่ใจว่า request นั้นถูกส่งมาจาก Client จริง\end{tabular}                                        \\ \hline
    5           & \begin{tabular}[c]{@{}l@{}}เนื่องจากตัวเกมยังไม่สมบูรณ์ \\ จึงสามารถทำ AI ได้เพียงบางส่วนเท่านั้น\end{tabular}                                                                                                & \begin{tabular}[c]{@{}l@{}}AI ของเกมจะถูกเทรนด้วย Task ที่เล็กลงแทน \\ เช่นการวิ่งหาน้ำตาลภายในแผนที่ \\ และนำไปสะสมที่ฐานให้ครบตามกำหนด\end{tabular}                                                                   \\ \hline
    6           & \begin{tabular}[c]{@{}l@{}}Unity ปรับเปลี่ยน Version Control จาก Unity Collaboration \\ เป็น Plastic SCM ทำให้งานบางส่วนไม่ถูกบันทึกไว้และหายไป\end{tabular}                                                  & เปลี่ยนไปใช้ Plastic SCM และทำงานในส่วนที่หายไปขึ้นใหม่                                                                                                                                                                 \\ \hline
    \end{tabular}
  \end{table}

\section{สรุปผลการดำเนินงาน} 
จากการพัฒนาเกมตามที่ออกแบบไว้ในบทที่ 3 สามารถสรุปผลการพัฒนาได้ ดังนี้

\begin{enumerate}
  \item \textbf{Tutorial}
  
  เมื่อกดปุ่ม Tutorial เข้ามาแล้วจะปรากฎบทพูดของตัวละครต่าง ๆ ดังรูปที่~\ref{fig:4-1}
  โดยใน Tutorial จะสอนผู้เล่นเกี่ยวกับส่วนต่าง ๆ ของเกม พร้อมทั้งแทรกความรู้เรื่อง PDCA 
  โดยสิ่งที่จะสอนผู้เล่น มีดังต่อไปนี้
  \begin{itemize}
    \item วิธีการจัด Unit เข้าทีม เพื่อไปสู้กับฝ่ายตรงข้ามในหน้า Lobby 
    \item การเดินหรือเคลื่อนที่ของตัวละครในเกมโดยการคลิกซ้ายไปยังตำแหน่งที่ต้องการ
    \item การสร้าง tower เพื่อป้องกันหมู่บ้าน
    \item การเก็บน้ำตาลโดยการเดินเข้าไปใกล้ ๆ น้ำตาลที่อยู่บนพื้น
    \item การฝากน้ำตาลเข้าฐานเพื่อสะสมให้ถึงเงื่อนไขของการจบเกม
  \end{itemize}

  หลังจากจบการเล่น Tutorial แล้วจะปรากฏหน้า Result ขึ้นตอนจบเกม
  และผู้เล่นก็จะได้รับรางวัลเป็นการตอบแทนดังรูปที่~\ref{fig:4-3} หลังจากที่ผู้เล่นกดปุ่ม OK 
  ผู้เล่นก็จะถูกนำกลับไปที่หน้า Home

  \begin{figure}[H]\centering
    \includegraphics[width=10cm]{./imgs/4-1.png}
    \caption{ตัวอย่างบทพูดของตัวละครใน tutorial}\label{fig:4-1}
  \end{figure}

  \begin{figure}[H]\centering
    \includegraphics[width=10cm]{./imgs/4-2.png}
    \caption{ตัวอย่างการแทรกความรู้เรื่อง PDCA}\label{fig:4-2}
  \end{figure}

  \begin{figure}[H]\centering
    \includegraphics[width=10cm]{./imgs/4-3.png}
    \caption{หน้า Result}\label{fig:4-3}
  \end{figure}

  \item \textbf{Player Controller}
    การบังคับตัวละครภายในเกม จะใช้เมาส์คลิกซ้ายไปยังตำแหน่งที่ต้องการ 
    และคลิกขวา เพื่อกดใช้งานหรือเปิดเมนูจากวัตถุภายในแผนที่ ซึ่งภายในเกม
    ผู้เล่นจะเห็นตำแหน่งของผู้เล่นอีกคนหนึ่ง โดยใช้ Network Object Component 
    ประกอบกับ Network Transform Component เพื่อ Synchronize ข้อมูลตำแหน่งของผู้เล่นแต่ละคน

  \item \textbf{Player Sugar}
    ผู้เล่นสามารถเก็บน้ำตาลภายในแผนที่ได้ ซึ่งจำนวนน้ำตาลจะแสดงออกมาเป็นหลอดน้ำตาลใต้ตัวละครของผู้เล่น 
    โดยมีการใช้  Network Variable เก็บตัวเลขน้ำตาลของผู้เล่นแต่ละคนเอาไว้ เมื่อจำนวนน้ำตาลเปลี่ยนไป 
    หลอดน้ำตาลจะรับคำสั่งเพื่ออัพเดทจำนวนน้ำตาลที่ควรแสดง 
% //TODO
    % \begin{figure}[H]\centering
    %   \includegraphics[width=10cm]{./imgs/4-4.png}
    %   \caption{ผู้เล่นที่มีหลอดน้ำตาล}\label{fig:4-4}
    % \end{figure}

  \item \textbf{Player Animation}
    ตัวละครมี Animation เพื่อแสดงถึงการกระทำต่าง ๆ ของผู้เล่น 
    ได้แก่ การเดิน การเก็บน้ำตาล และการสร้าง Unit ซึ่ง Animation 
    ภายในเกมจะถูกควบคุมด้วย Animator Controller ที่สามารถสร้าง parameter 
    และเขียนโปรแกรมปรับ parameter ให้เปลี่ยน Animation ของตัวละคร 

    Animation ของตัวละคร ใช้วิธีการวาดส่วนต่างๆ ของตัวละคร แล้วนำไปประกอบ
    เป็นตัวละครในโปรแกรม DragonBones เพื่อทำ Animation และ Export 
    ออกมาเป็น Sprite Sheet แบบ frame-by-frame
    % //TODO
    % \begin{figure}[H]\centering
    %   \includegraphics[width=10cm]{./imgs/4-5.png}
    %   \caption{Player Animation}\label{fig:4-5}
    % \end{figure}
    
  \item \textbf{Interactable Object}
    ภายในแผนที่มีวัตถุต่าง ๆ ที่จัดวางภายในแผนที่เพื่อให้ผู้เล่นกดใช้งาน 
    เช่น พุ่มน้ำตาล หรือเหมืองน้ำตาล เมื่อผู้เล่นกดใช้งานวัตถุดังกล่าว 
    จะประกาศ Event ที่ชื่อ OnInteract ออกไป และฟังก์ชันที่ Subscribe Event 
    ดังกล่าวจะทำงาน ด้วยวิธีนี้จะสามารถแยก Module ออกมาเพื่อให้แต่ละ Module ทำงานแยกออกจากกันได้
    % //TODO
    % \begin{figure}[H]\centering
    %   \includegraphics[width=10cm]{./imgs/4-6.png}
    %   \caption{ตัวอย่าง Interactable Object}\label{fig:4-6}
    % \end{figure}
    
  \item \textbf{User Interface Design}
    ผู้จัดทำได้ออกแบบหน้าจอภายในเกมโดยเริ่มจากการวาด asset 
    ด้วยโปรแกรม Adobe Illustrator และ Export ออกมาเป็นภาพ 
    หลังจากนั้นก็นำ asset ต่าง ๆ มาจัดวางบน Canvas ใน Unity 
    ซึ่งได้ออกแบบมาเป็นหน้าต่าง ๆ ได้แก่

    \begin{itemize}
      \item \textbf{หน้า Home} มีส่วนประกอบของหน้า ดังนี้ 
      \begin{enumerate}
        \item \textbf{ปุ่ม Bag} ที่เมื่อกดแล้วจะถูกนำทางไปที่หน้า Inventory 
        \item \textbf{ปุ่ม Play} ที่เมื่อกดแล้วจะเข้าสู่การเล่นเกม
      \end{enumerate}

      \begin{figure}[H]\centering
        \includegraphics[width=10cm]{./imgs/4-7.png}
        \caption{หน้า Home}\label{fig:4-7}
      \end{figure}
      
      
      \item \textbf{หน้าภายในเกม} มีส่วนประกอบของหน้า ดังนี้ 
      \begin{enumerate}
        \item \textbf{นาฬิกา} ที่จะแสดงเวลาที่เหลือภายในเกมตรงบนกึ่งกลางจอ
        \item \textbf{หลอดความคืบหน้าการเก็บน้ำตาลของแต่ละฝ่าย} ได้แก่ หลอดสีแดงของ Attacker และหลอดสีฟ้าของ Defender
        \item \textbf{ปุ่ม setting} ที่เป็นรูปเฟืองเมื่อกดจะปรากฏหน้าต่างการตั้งค่าขึ้นมา
        \item \textbf{UI แสดงสกิลตัวละคร} ที่จะแสดงสกิลที่ตัวละครสวมใส่อยู่และคูลดาวน์หลังจากใช้สกิลนั้น ๆ
        
        \begin{figure}[H]\centering
            \includegraphics[width=10cm]{./imgs/4-8.png}
            \caption{นาฬิกาและหลอดความคืบหน้าการเก็บน้ำตาล}\label{fig:4-8}
          \end{figure}
          \begin{figure}[H]\centering
            \begin{minipage}{.3\textwidth}
                \centering
                \includegraphics[width=1.5cm]{./imgs/4-9.png}
                \caption{ปุ่ม Setting}\label{fig:4-9}
            \end{minipage}
            \begin{minipage}{.3\textwidth}
                \centering
                \includegraphics[width=2cm]{./imgs/4-10.png}
                \caption{UI แสดงสกิลตัวละคร}\label{fig:4-10}
            \end{minipage}
          \end{figure}
      \end{enumerate}         

      \item \textbf{หน้า Inventory} มีส่วนประกอบของหน้า ดังนี้ 
      \begin{enumerate}
        \item \textbf{ปุ่ม Back} ที่เมื่อกดแล้วจะถูกนำทางไปที่หน้า Inventory 
        \item \textbf{แท็บแยกประเภทไอเทมภายในเกม} ที่เมื่อกดแล้วจะเข้าสู่การเล่นเกม
        \item \textbf{หน้าต่างแสดงไอเทม} ที่แสดงไอเทมตามประเภทที่กดเลือกไว้ตรงแท็บ
        \item \textbf{Panel} ที่แสดงรายละเอียดไอเทมที่เลือกไว้
      \end{enumerate}
      
      \begin{figure}[H]\centering
        \includegraphics[width=10cm]{./imgs/4-11.png}
        \caption{หน้า Inventory}\label{fig:4-11}
      \end{figure}

      \item \textbf{หน้า Lobby} มีส่วนประกอบของหน้า ดังนี้ 
      \begin{enumerate}
        \item \textbf{นาฬิกา}แสดงเวลาที่เหลือภายในขั้นตอนการจัดทีม
        \item \textbf{กล่องแสดง Unit ที่ฝ่ายตรงข้ามเลือก} 
        \item \textbf{กล่องที่จะแสดง Unit ที่เราได้เลือกไว้} โดยเมื่อกดกล่องนี้จะปรากฏหน้า เลือก Unit 
        \item \textbf{ปุ่ม Ready} ที่แสดงสถานะพร้อมและรอเริ่มเกม โดยเมื่อทั้งสองฝ่ายพร้อมเกมก็จะเริ่ม
      \end{enumerate}
      
      \begin{figure}[H]\centering
        \includegraphics[width=10cm]{./imgs/4-12.png}
        \caption{หน้า Lobby}\label{fig:4-12}
      \end{figure}
     
      \item \textbf{หน้า Result ตอนจบเกม} มีส่วนประกอบของหน้า ดังนี้ 
      \begin{enumerate}
        \item \textbf{ข้อความแสดงสถานะการแพ้ชนะเกม}
        \item \textbf{แท็บที่ใช้เลือกดู Rewards และ Scores ของตนเอง} 
        \item \textbf{หน้าต่างแสดงรายละเอียดสิ่งที่เลือกไว้ตรงแท็บ} 
        \item \textbf{ปุ่ม OK} ที่เมื่อกดจะกลับไปยังหน้า Home 
      \end{enumerate}

      \begin{figure}[H]\centering
        \includegraphics[width=10cm]{./imgs/4-13.png}
        \caption{หน้า Result}\label{fig:4-13}
      \end{figure}
      
      \item \textbf{หน้าร้านค้าภายในเกม} มีส่วนประกอบของหน้า ดังนี้ 
      \begin{enumerate}
        \item \textbf{แท็บเลือกประเภทสิ่งที่จะซื้อหรืออัพเกรด} ได้แก่ Unit และ Skill 
        \item \textbf{ตัวเลข} ที่บอกจำนวนน้ำตาลที่ผู้เล่นมีอยู่ 
        \item \textbf{ปุ่ม X} ที่เมื่อกดจะปิดหน้าต่างร้านค้านี้ 
        \item \textbf{Card} สิ่งของที่สามารถซื้อหรืออัพเกรดได้โดยภายใน Card จะประกอบด้วย เลเวล 
        ภาพของ Unit หรือ Skill และปุ่มที่บอกราคาน้ำตาลที่ต้องจ่ายเมื่ออัพเกรดหรือซื้อ 
      \end{enumerate}

      \begin{figure}[H]\centering
        \includegraphics[width=10cm]{./imgs/4-14.png}
        \caption{หน้าร้านค้าภายในเกม}\label{fig:4-14}
      \end{figure}

    \end{itemize}

\end{enumerate}

\section{การสอบถามความพึงพอใจจากผู้ใช้งาน}

%%%%%%%%%%%%%%%%%%%%%%%%%%%%%%%%%%%%%%%%%%%%%%%%%%%%%%%%%%%%%%%
%%%%%%%%%%%%%%%%%%%% Conclusions %%%%%%%%%%%%%%%%%%%%%%%%%%%%%
%%%%%%%%%%%%%%%%%%%%%%%%%%%%%%%%%%%%%%%%%%%%%%%%%%%%%%%%%%%%%%%
% \chapter{บทสรุป}

% This chapter is optional for proposal and progress reports but 
% is required for the final report.

% \section{สรุปผลโครงงาน}
% สรุปว่าโครงงานบรรลุตามวัตถุประสงค์ที่ตั้งไว้หรือไม่ อย่างไร 

% \section{ปัญหาที่พบและการแก้ไข}
% State your problems and how you fixed them.

% \section{ข้อจำกัดและข้อเสนอแนะ}
% ข้อจำกัดของโครงงาน What could be done in the future to make your projects better.

%%%%%%%%%%%%%%%%%%%%%%%%%%%%%%%%%%%%%%%%%%%%%%%%%%%%%%%%%%%%%%%
%%%%%%%%%%%%%%%%%%%% Bibliography %%%%%%%%%%%%%%%%%%%%%%%%%%%%%
%%%%%%%%%%%%%%%%%%%%%%%%%%%%%%%%%%%%%%%%%%%%%%%%%%%%%%%%%%%%%%%

%%%% Comment this in your report to show only references you have
%%%% cited. Otherwise, all the references below will be shown.
%\nocite{*}
%% Use the kmutt.bst for bibtex bibliography style 
%% You must have cpe.bib and string.bib in your current directory.
%% You may go to file .bbl to manually edit the bib items.

\makeatletter
\g@addto@macro{\UrlBreaks}{\UrlOrds}
\makeatother

\bibliographystyle{kmutt}
\bibliography{string,cpe}

%%%%%%%%%%%%%%%%%%%%%%%%%%%%%%%%%%%%%%%%%%%%%%%%%%%%%%%%%%%%%%%
%%%%%%%%%%%%%%%%%%%%%%%% Appendix %%%%%%%%%%%%%%%%%%%%%%%%%%%%%
%%%%%%%%%%%%%%%%%%%%%%%%%%%%%%%%%%%%%%%%%%%%%%%%%%%%%%%%%%%%%%%
% \appendix{รายละเอียดเกมเพิ่มเติม}
% \noindent{\large\bf Player Skills} มี 2 ประเภท คือ

% \begin{enumerate}
%   \item \textbf{Passive} สกิลที่ผู้เล่นไม่ต้องกดใช้งาน ได้แก่
%   \begin{itemize}
%     \item \textbf{Discount :} 
%     \item \textbf{Vision :} 
%     \item \textbf{Movement Speed :} 
%   \end{itemize}
%   \item \textbf{Active} สกิลที่ผู้เล่นต้องกดใช้งาน ได้แก่
%   \begin{itemize}
%     \item \textbf{Heal :} 
%     \item \textbf{Buff SPD :} 
%     \item \textbf{Buff ATK :} 
%     \item \textbf{Dash :} 
%     \item \textbf{Footprint Tracker :} 
%   \end{itemize}
% \end{enumerate}

% \noindent{\large\bf Unit/Tower} ทั้ง Unit และ Tower นั้นแบ่งออกเป็น 
% 3 ประเภทหลัก ๆ ดังนี้

% \begin{enumerate}
%   \item \textbf{Attack :} Unit/Tower นี้จะมีหน้าที่โจมตีเป็นหลัก ได้แก่
%   \begin{table}[H]
%     \begin{tabular}{|l|ll|}
%     \hline
%     \rowcolor[HTML]{C0C0C0} 
%     \multicolumn{1}{|c|}{\cellcolor[HTML]{C0C0C0}\textbf{Name}} & \multicolumn{1}{c|}{\cellcolor[HTML]{C0C0C0}\textbf{Unit}} & \multicolumn{1}{c|}{\cellcolor[HTML]{C0C0C0}\textbf{Tower}} \\ \hline
%     Bread                                                       & \multicolumn{2}{c|}{-Basic Unit-}                                                                                        \\ \hline
%     Pudding                                                     & \multicolumn{1}{l|}{\%Stun Tower}                          & Slow Unit                                                   \\ \hline
%     Taiyaki                                                     & \multicolumn{1}{l|}{เคลื่อนที่ได้เร็ว มีเลือดน้อย}                  & โจมตีเป็นวงกว้างแต่ทำดาเมจได้น้อย                                                      \\ \hline
%     Chocolate Lava Cake                                         & \multicolumn{1}{l|}{สามารถโจมตี Tower ได้}                   & ทำดาเมจได้มาก                                                      \\ \hline
%     \end{tabular}
%     \end{table}

%   \item \textbf{Defend :} Unit/Tower นี้จะมีหน้าที่ป้องกันเป็นหลัก ได้แก่
  

%   \item \textbf{Enchanter :} Unit/Tower นี้จะมี Skill ที่สามารถช่วยเหลือ 
%   Unit/Tower อื่นๆ ได้ ได้แก่



% \end{enumerate}

%%%%%%%%%%%%%%%%%%%%%%%%%%%%%%%%%%%%%%%%%%%%%%%%%%%%%%%%%%
%%%%%%%%%%%%%%% The 2nd appendix %%%%%%%%%%%%%%%%%%%%%%%%%%
%%%%%%%%%%%%%%%%%%%%%%%%%%%%%%%%%%%%%%%%%%%%%%%%%%%%%%%%%%
% \appendix{ชื่อภาคผนวกที่ 2}
% \noindent{\large\bf ใส่หัวข้อตามความเหมาะสม} \\

% Next, we show how $\mathrm{Var}\{X_n\}$ can be determined.  Let
% $C_{\lambda}(l)$ be the autocovariance function of $\lambda_n$.  The
% MVA technique basically approximates the input process $\lambda_n$
% with a Gaussian process, which allows $\mathrm{Var}\{X_n\}$ to be
% represented by the autocovariance function.  In particular, the
% variance of $X_n$ can be expressed in terms of $C_{\lambda}(l)$ as
% \begin{equation}
%   \mathrm{Var}\{X_n\} = nC_{\lambda}(0) + 2\sum_{l=1}^{n-1} (n-l)C_{\lambda}(l)
% \end{equation} 

% \noindent{\large\bf Add more topic as you need} \\

% Therefore, $C_{\lambda}(l)$ must be known in the MVA technique, either
% by assuming specific traffic models or by off-line analysis in case of
% traces.  In most practical situations, $C_{\lambda}(l)$ will not be
% known in advance, and an on-line measurement algorithm developed
% in~\cite{eun01} is required to jointly determine both $n^\ast$ and
% $m_x$. For fGn traffic, $\mathrm{Var}\{X_n\}$ is equal to $\sigma^2
% n^{2H}$, where $\sigma^2 = \mathrm{Var}\{\lambda_n\}$, and we can find
% the $n^\ast$ that minimizes (\ref{eq:mx}) directly. Although $\lambda$
% can be easily measured, it is not the case for $\sigma^2$ and $H$.
% Consequently, the MVA technique suffers from the need of prior
% knowledge traffic parameters. 





\end{document}
