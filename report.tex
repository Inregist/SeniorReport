%%%%%% Run at command line, run
%%%%%% xelatex grad-sample.tex 
%%%%%% for a few times to generate the output pdf file
\documentclass[12pt,oneside,openright,a4paper]{cpe-thai-project}


\usepackage{polyglossia}
\setdefaultlanguage{thai}
\setotherlanguage{english}
\newfontfamily\thaifont[Script=Thai,Scale=1.23]{TH Sarabun New}
\defaultfontfeatures{Mapping=tex-text,Scale=1.23,LetterSpace=0.0}
\setmainfont[Scale=1.23,LetterSpace=0,WordSpace=1.0,FakeStretch=1.0,Mapping=tex-text]{TH Sarabun New}
\XeTeXlinebreaklocale "th"	
\XeTeXlinebreakskip = 0pt plus 0pt
\emergencystretch=10pt

%%%%%%%%%%%%%%%%%%%%%%%%%%%%%%%%%%%%%%%%%%%%%%%%%%%%%%%%%%%%%%%%%%%
% Customize below to suit your needs 
% The ones that are optional can be left blank. 
%%%%%%%%%%%%%%%%%%%%%%%%%%%%%%%%%%%%%%%%%%%%%%%%%%%%%%%%%%%%%%%%%%%
% First line of title
\def\disstitleone{Sweets vs Sweets :}   
% Second line of title
\def\disstitletwo{Real Time Strategy Multiplayer Tower Defense game}   
% Your first name and lastname
\def\dissauthor{Mr. Chirateep Pakdeengam}   % 1st member
%%% Put other group member names here ..
\def\dissauthortwo{Ms. Premyuda Angkawichai}   % 2nd member (optional)
\def\dissauthorthree{}   % 3rd member (optional)

% The degree that you're persuing..
\def\dissdegree{Bachelor of Engineering} % Name of the degree
\def\dissdegreeabrev{B.Eng} % Abbreviation of the degree
\def\dissyear{2021}                   % Year of submission
\def\thaidissyear{2564}               % Year of submission (B.E.)

%%%%%%%%%%%%%%%%%%%%%%%%%%%%%%%%%%%%%%%%%%%%
% Your project and independent study committee..
%%%%%%%%%%%%%%%%%%%%%%%%%%%%%%%%%%%%%%%%%%%%
\def\dissadvisor{Assoc.Prof. Dr.Natasha Dejdumrong, D.Tech.Sci.}  % Advisor
%%% Leave it empty if you have no Co-advisor
\def\disscoadvisor{}  % Co-advisor
\def\disscommitteetwo{Asst.Prof. Dr.Nuttanart Facundes, Ph.D.}  % 3rd committee member (optional)
\def\disscommitteethree{Asst.Prof. Dr.Khajonpong Akkarajitsakul, Ph.D.}   % 4th committee member (optional) 
\def\disscommitteefour{Assoc.Prof. Dr.Naruemon Wattanapongsakorn, Ph.D.}    % 5th committee member (optional) 

\def\worktype{Project} %%  Project or Independent study
%//TODO:
\def\disscredit{3}   %% 3 credits or 6 credits

\def\fieldofstudy{Computer Engineering} 
\def\department{Computer Engineering} 
\def\faculty{Engineering}

\def\thaifieldofstudy{วิศวกรรมคอมพิวเตอร์} 
\def\thaidepartment{วิศวกรรมคอมพิวเตอร์} 
\def\thaifaculty{วิศวกรรมศาสตร์}
 
\def\appendixnames{Appendix} %%% Appendices or Appendix

\def\thaiworktype{ปริญญานิพนธ์} %  Project or research project % 
\def\thaidisstitleone{Sweets vs Sweets :}
\def\thaidisstitletwo{Real Time Strategy Multiplayer Tower Defense game}
\def\thaidissauthor{นายจิรทีปต์ ภักดีงาม}
\def\thaidissauthortwo{นางสาวเปรมยุดา อังคะวิชัย} %Optional
\def\thaidissauthorthree{} %Optional

\def\thaidissadvisor{รศ.ดร.ณัฐชา เดชดำรง}
%% Leave this empty if you have no co-advisor
\def\thaidisscoadvisor{} %Optional
\def\thaidissdegree{วิศวกรรมศาสตรบัณฑิต}

% Change the line spacing here...
\linespread{1.15}

%%%%%%%%%%%%%%%%%%%%%%%%%%%%%%%%%%%%%%%%%%%%%%%%%%%%%%%%%%%%%%%%
% End of personal customization.  Do not modify from this part 
% to \begin{document} unless you know what you are doing...
%%%%%%%%%%%%%%%%%%%%%%%%%%%%%%%%%%%%%%%%%%%%%%%%%%%%%%%%%%%%%%%%


%%%%%%%%%%%% Dissertation style %%%%%%%%%%%
%\linespread{1.6} % Double-spaced  
%%\oddsidemargin    0.5in
%%\evensidemargin   0.5in
%%%%%%%%%%%%%%%%%%%%%%%%%%%%%%%%%%%%%%%%%%%
%\renewcommand{\subfigtopskip}{10pt}
%\renewcommand{\subfigbottomskip}{-5pt} 
%\renewcommand{\subfigcapskip}{-6pt} %vertical space between caption
%                                    %and figure.
%\renewcommand{\subfigcapmargin}{0pt}

\renewcommand{\topfraction}{0.85}
\renewcommand{\textfraction}{0.1}

\newtheorem{theorem}{Theorem}
\newtheorem{lemma}{Lemma}
\newtheorem{corollary}{Corollary}

\def\QED{\mbox{\rule[0pt]{1.5ex}{1.5ex}}}
\def\proof{\noindent\hspace{2em}{\itshape Proof: }}
\def\endproof{\hspace*{\fill}~\QED\par\endtrivlist\unskip}
%\newenvironment{proof}{{\sc Proof:}}{~\hfill \blacksquare}
%% The hyperref package redefines the \appendix. This one 
%% is from the dissertation.cls
%\def\appendix#1{\iffirstappendix \appendixcover \firstappendixfalse \fi \chapter{#1}}
%\renewcommand{\arraystretch}{0.8}
%%%%%%%%%%%%%%%%%%%%%%%%%%%%%%%%%%%%%%%%%%%%%%%%%%%%%%%%%%%%%%%%
%%%%%%%%%%%%%%%%%%%%%%%%%%%%%%%%%%%%%%%%%%%%%%%%%%%%%%%%%%%%%%%%

\usepackage{ragged2e}
\begin{document}

\pdfstringdefDisableCommands{%
\let\MakeUppercase\relax
}

\begin{center}
  \includegraphics[width=2.8cm]{imgs/logo02.jpg}
\end{center}
\vspace*{-1cm}

\maketitlepage
\makesignaturepage 

%%%%%%%%%%%%%%%%%%%%%%%%%%%%%%%%%%%%%%%%%%%%%%%%%%%%%%%%%%%%%%
%%%%%%%%%%%%%%%%%%%%%% English abstract %%%%%%%%%%%%%%%%%%%%%%%
%%%%%%%%%%%%%%%%%%%%%%%%%%%%%%%%%%%%%%%%%%%%%%%%%%%%%%%%%%%%%%

\abstract

Resource management, Planning and Decision making are fundamental skills 
that everyone should learn. But it is too risky to learn these skills in real-life situations.

We would like to create a Real-time multiplayer Tower Defense game for people to practice 
those skills mentioned above with risk-free. Players can challenge others to fight in the 
world of delicious sweets and candy. Our target group are people who love to play games such as 
students between the ages of 15-24 years old. The game can be played on MacOS and Windows PC.

\begin{flushleft}
\begin{tabular*}{\textwidth}{@{}lp{0.8\textwidth}}
\textbf{Keywords}: & Real Time Strategy Game / Tower Defense / Multiplayer Game / 2.5D Isometric Game / Machine Learning
\end{tabular*}
\end{flushleft}
\endabstract

%%%%%%%%%%%%%%%%%%%%%%%%%%%%%%%%%%%%%%%%%%%%%%%%%%%%%%%%%%%%%%
%%%%%%%%%% Thai abstract here %%%%%%%%%%%%%%%%%%%%%%%%%%%%%%%%%
%%%%%%%%%%%%%%%%%%%%%%%%%%%%%%%%%%%%%%%%%%%%%%%%%%%%%%%%%%%%%%
% {\newfontfamily\thaifont{TH Sarabun New:script=thai}[Scale=1.3]
% \XeTeXlinebreaklocale "th_TH"	
% \thaifont

\thaiabstract

การจัดการ การวางแผน และการตัดสินใจเป็นทักษะพื้นฐานที่ทุกคนควรมี 
แต่ถ้าเรียนรู้และฝึกทักษะดังกล่าวในสถานการณ์จริงจะค่อนข้างมีความเสี่ยง เช่น การฝึกวางแผนการลงทุนโดยใช้เงินจริง 
มีโอกาสเสี่ยงสูงในการขาดทุน 

ผู้จัดทำจึงต้องการพัฒนาเกมที่สามารถให้ผู้เล่นได้ฝึกทัศนคติการวางแผน 
ทักษะการตัดสินใจ และการแก้ไขปัญหาเฉพาะหน้าได้โดยปลอดภัย ไร้ความเสี่ยง 
พร้อมสนุกไปกับการเรียนรู้ โดยเป็นเกม Tower Defense ที่มีการต่อสู้กันระหว่างผู้เล่นสองคนแบบ 
real-time multiplayer จำลองเหตุการณ์ในเกมให้อยู่ในโลกของขนมหวานที่น่าดึงดูดใจ 
โดยมีกลุ่มเป้าหมายคือกลุ่มคนที่ชื่นชอบการเล่นเกม ได้แก่ นักเรียนนักศึกษาที่อยู่ในช่วงอายุ 15-24 ปี 
สามารถเล่นได้ทั้งบนระบบปฏิบัติการ Windows และ MacOS


\begin{flushleft}
\begin{tabular*}{\textwidth}{@{}lp{0.8\textwidth}}
 & \\

\textbf{คำสำคัญ}: & Real Time Strategy Game / Tower Defense / Multiplayer Game / 2.5D Isometric Game / Machine Learning
\end{tabular*}
\end{flushleft}
\endabstract

%}

%%%%%%%%%%%%%%%%%%%%%%%%%%%%%%%%%%%%%%%%%%%%%%%%%%%%%%%%%%%%
%%%%%%%%%%%%%%%%%%%%%%% Acknowledgments %%%%%%%%%%%%%%%%%%%%
%%%%%%%%%%%%%%%%%%%%%%%%%%%%%%%%%%%%%%%%%%%%%%%%%%%%%%%%%%%%

\preface
โครงงานนี้สำเร็จลงได้ด้วยความช่วยเหลืออย่างดียิ่งจาก รศ.ดร.ณัฐชา เดชดำรง ที่ปรึกษาโครงงาน 
ที่กรุณาสละเวลาให้ความรู้ คำปรึกษา คำแนะนำ และข้อเสนอแนะที่เป็นประโยชน์อย่างมาก 
อีกทั้งยังคอยติดตามดูแลเอาใจใส่ตลอดการทำโครงงานนี้จนสำเร็จลุล่วงได้ด้วยดี 
ผู้จัดทำโครงงานจึงขอกราบขอบพระคุณเป็นอย่างสูงไว้ ณ ที่นี้ด้วย

สุดท้ายนี้ขอขอบคุณเพื่อน ๆ พี่ ๆ และ น้อง ๆ ในภาควิชาวิศวกรรมคอมพิวเตอร์ทุกคนที่คอยให้ความช่วยเหลือเป็นอย่างดี


%%%%%%%%%%%%%%%%%%%%%%%%%%%%%%%%%%%%%%%%%%%%%%%%%%%%%%%%%%%%%
%%%%%%%%%%%%%%%% ToC, List of figures/tables %%%%%%%%%%%%%%%%
%%%%%%%%%%%%%%%%%%%%%%%%%%%%%%%%%%%%%%%%%%%%%%%%%%%%%%%%%%%%%
% The three commands below automatically generate the table 
% of content, list of tables and list of figures
\tableofcontents                    
\listoftables
\listoffigures                      

%%%%%%%%%%%%%%%%%%%%%%%%%%%%%%%%%%%%%%%%%%%%%%%%%%%%%%%%%%%%%%
%%%%%%%%%%%%%%%%%%%%% List of symbols page %%%%%%%%%%%%%%%%%%%
%%%%%%%%%%%%%%%%%%%%%%%%%%%%%%%%%%%%%%%%%%%%%%%%%%%%%%%%%%%%%%
% You have to add this manually..
% \listofsymbols
% \begin{flushleft}
% \begin{tabular}{@{}p{0.07\textwidth}p{0.7\textwidth}p{0.1\textwidth}}
% \textbf{SYMBOL}  & & \textbf{UNIT} \\[0.2cm]
% $\alpha$ & Test variable\hfill & m$^2$ \\
% $\lambda$ & Interarival rate\hfill &  jobs/second\\
% $\mu$ & Service rate\hfill & jobs/second\\
% \end{tabular}
% \end{flushleft}
%%%%%%%%%%%%%%%%%%%%%%%%%%%%%%%%%%%%%%%%%%%%%%%%%%%%%%%%%%%%%%
%%%%%%%%%%%%%%%%%%%%% List of vocabs & terms %%%%%%%%%%%%%%%%%
%%%%%%%%%%%%%%%%%%%%%%%%%%%%%%%%%%%%%%%%%%%%%%%%%%%%%%%%%%%%%%
% You also have to add this manually..
% \listofvocab
% \begin{flushleft}
% \begin{tabular}{@{}p{1in}@{=\extracolsep{0.5in}}l}
% Test &  test \\
% MANET & Mobile Ad Hoc Network 
% \end{tabular}
% \end{flushleft}

%\setlength{\parskip}{1.2mm}

%%%%%%%%%%%%%%%%%%%%%%%%%%%%%%%%%%%%%%%%%%%%%%%%%%%%%%%%%%%%%%%
%%%%%%%%%%%%%%%%%%%%%%%% Main body %%%%%%%%%%%%%%%%%%%%%%%%%%%%
%%%%%%%%%%%%%%%%%%%%%%%%%%%%%%%%%%%%%%%%%%%%%%%%%%%%%%%%%%%%%%%


\chapter{บทนำ}

\section{ที่มาและความสำคัญ}

การจัดการ การวางแผน และการตัดสินใจเป็นทักษะพื้นฐานที่ทุกคนควรมี 
การเรียนรู้ทักษะเหล่านี้ ผู้เรียนรู้ต้องลองผิดลองถูก หาข้อผิดพลาด พัฒนาปรับปรุงวิธีการคิด 
การจัดการ และการตัดสินใจของตนเอง แต่ถ้าเรียนรู้และฝึกทักษะดังกล่าวในสถานการณ์จริง 
จะค่อนข้างมีความเสี่ยง เช่น การฝึกวางแผนการลงทุนโดยใช้เงินจริง มีโอกาสเสี่ยงสูงในการขาดทุน 
ดังนั้น หากมีพื้นที่ทดลองฝึกทักษะดังกล่าวโดยที่ไม่มีความเสี่ยง (Sandbox) ผู้
เรียนจะสามารถเรียนรู้ได้อย่างไม่ต้องเป็นกังวล

การเรียนรู้จากเกมเป็นวิธีการเรียนรู้แบบหนึ่งที่ค่อนข้างมีประสิทธิภาพ และไม่มีความเสี่ยง 
สามารถทำให้ผู้ที่เล่นเกมได้สนุกไปกับการเรียนรู้ หรือได้เรียนรู้บางอย่างจากเกมโดยไม่รู้ตัว 
และเกิดการเรียนรู้ได้ง่าย แต่การออกแบบเกมให้สามารถเรียนรู้ได้อย่างสนุกสนานนั้นไม่ใช่เรื่องง่าย 
ผู้ออกแบบเกมต้องคิดและออกแบบเกมให้ผู้เล่นสามารถสนุกสนาน พร้อมกับแทรกเนื้อหาสาระหรือทักษะบางอย่างเข้าไปในเกม 
โดยไม่ให้ผู้เล่นรู้สึกว่าเนื้อหาแน่นหรือน่าเบื่อเกินไป

ผู้จัดทำอยากพัฒนาเกมที่สามารถให้ผู้เล่นได้ฝึกทัศนคติการวางแผน ทักษะการตัดสินใจ และการแก้ไขปัญหาเฉพาะหน้า 
โดยเป็นการต่อสู้กันระหว่างผู้เล่นสองคนแบบ real-time multiplayer และมีการเก็บข้อมูลการเล่นของผู้เล่น 
เพื่อนำมาพัฒนา AI จาก Machine Learning ที่สามารถเลียนแบบการเล่นให้เหมือนมนุษย์ได้

สาเหตุที่ผู้จัดทำพัฒนาเกมในรูปแบบ real-time multiplayer เนื่องจากการที่ผู้เล่นได้สู้กับผู้เล่นจริงนั้น
ทำให้สิ่งที่เกิดขึ้นภายในเกมมีความสดใหม่ ไม่ซ้ำเดิม เพิ่ม Replayability (คุณค่าของการเล่นเกมเดิมซ้ำ) 
เพราะผู้เล่นแต่ละคนมีความคิดหรือการวางแผนที่แตกต่างกัน และการสร้าง AI จาก Machine Learning 
สามารถนำมาเพื่อวัดระดับของผู้เล่นได้ด้วยการปรับระดับความยากของ AI ทำให้เราสามารถประเมินระดับการเล่นของผู้เล่นได้ 
เพื่อเป็นตัวชี้วัดว่าผู้เล่นสามารถพัฒนาทักษะที่กล่าวไปข้างต้นได้ผ่านการเล่นเกม


\section{วัตถุประสงค์}

\begin{itemize}
\item เพื่อให้ผู้เล่นได้ฝึกกระบวนการคิดวางแผนแบบ Real Time 
\item เพื่อให้ผู้เล่นได้ฝึกทักษะการแก้ไขปัญหาเฉพาะหน้า
\item เพื่อศึกษาและพัฒนาเกมด้วย Unity โดยใช้ภาษา C\#
\end{itemize}


\section{ขอบเขตของโครงงาน}

\begin{itemize}
\item  เกม Tower Defense ที่ผสมผสานกับ Real Time Strategy
\item  สามารถเล่นกับผู้เล่นอื่นได้ผ่านระบบ Multiplayer 
\item  สามารถเล่นได้ผ่านระบบปฏิบัติการ Windows และ MacOS
\item  พัฒนาเกมโดยใช้ Unity Engine และ C\# Programming Language 
\item  สร้าง AI เพื่อเลียนแบบการเล่นของผู้เล่น 
\end{itemize}


\section{ประโยชน์ที่คาดว่าจะได้่รับ}

เกมที่ผู้เล่นเล่นได้อย่างสนุกสนานได้ฝึกทักษะการวางแผน การบริหารจัดการ และการตัดสินใจ

% //TODO: gantt chart
\section{ขั้นตอนการดำเนินงาน}
\textbf{ภาคการศึกษาที่ 1}

\textbf{ภาคการศึกษาที่ 2}


\section{ผลการดำเนินงาน}

\textbf{ภาคการศึกษาที่ 1}
\begin{enumerate}
  \item ข้อเสนอโครงงาน
  \item เอกสารโครงร่างการออกแบบเกม (Game Design Document) 
  \item วิดีโอตัวอย่าง Gameplay
  \item Prototype เกม
  
\end{enumerate}

\textbf{ภาคการศึกษาที่ 2}
\begin{enumerate}
  \item เกมที่สามารถเล่นได้ตามที่ออกแบบไว้
  \item รายงานฉบับสมบูรณ์
  \item User Manual
\end{enumerate}

%%%%%%%%%%%%%%%%%%%%%%%%%%%%%%%%%%%%%%%%%%%%%%%%%%%%%%%%%%%%
%%%%%%%%%%%%%%  Literature Review %%%%%%%%%%%%%%%%%%%%%%%%%%
%%%%%%%%%%%%%%%%%%%%%%%%%%%%%%%%%%%%%%%%%%%%%%%%%%%%%%%%%%%%
\chapter{ทฤษฎีความรู้และงานที่เกี่ยวข้อง}

\section{บทนำ}
ในบทนี้ผู้จัดทำได้ศึกษาและหาข้อมูลเกี่ยวกับทฤษฎีที่เกี่ยวข้อง 
ภาษาหรือเครื่องมือที่จำเป็น และตัวอย่างเกมที่มีความคล้ายคลึงกับสิ่งที่จะศึกษา 
พร้อมทั้งพิจารณาข้อดีข้อเสีย และนำมาปรับใช้ สร้างความแตกต่างให้กับเกมที่จะพัฒนา


\section{ที่มาและความสำคัญ}

การจัดการ การวางแผน และการตัดสินใจเป็นทักษะพื้นฐานที่จำเป็น 
แต่การเรียนรู้ทักษะดังกล่าวต้องสามารถฝึกและเรียนรู้จากการลองผิดลองถูกได้โดยไม่มีความเสี่ยง 
การเรียนรู้จากเกมจึงเป็นวิธีการเรียนรู้ที่สามารถแก้ไขปัญหาดังกล่าวได้ 
โดยผู้เล่นสามารถเรียนรู้จากเกมที่จำลองออกมาเป็นสถานการณ์ให้ได้ฝึกคิดวางแผน 
แก้ปัญหาและตัดสินใจ พร้อมทั้งได้รับความสนุกในระหว่างการเรียนรู้ 
อีกทั้งยังสามารถแข่งขันกับผู้เล่นคนอื่นได้แบบ Real-time เพื่อเจอกับสถานการณ์ใหม่ที่ไม่ซ้ำเดิม 
และเพิ่มคุณค่าของการเล่นเกมเดิมซ้ำ (Replayability)

นอกจากนี้เพื่อให้สามารถวัดระดับทักษะของผู้เล่น ผู้จัดทำวางแผนพัฒนา AI ด้วย Machine Learning 
ที่สามารถเลียนแบบวิธีการเล่นของผู้เล่น และปรับระดับความยากของ AI 
เพื่อเป็นตัวชี้วัดว่าผู้เล่นสามารถพัฒนาทักษะที่กล่าวไปข้างต้นได้ผ่านการเล่นเกม


\section{ทฤษฎีที่เกี่ยวข้อง}

\subsection{Natural Funativity}
Natural Funativity เป็นทฤษฎีที่มีพื้นฐานมาจากแนวคิดว่าความสนุกต่าง ๆ 
ของมนุษย์มีรากฐานมาจากการล่าและเก็บสะสมสิ่งของตั้งแต่สมัยชนเผ่าในอดีต 
โดยความสนุกนั้นได้พัฒนามาเป็นความสนุกจากการฝึกฝนทักษะการเอาตัวรอด
และทักษะทางสังคมในปัจจุบัน สามารถแบ่งออกเป็น 3 ด้าน ดังนี้

\subsubsection{Physical Fun}
ความสนุกด้านการใช้ร่างกาย เช่น ความสนุกจากการเล่นกีฬา 
ความสนุกจากการสำรวจ ความสนุกจากใช้สายตาและมือร่วมกัน เป็นต้น
\subsubsection{Social Fun}
ความสนุกในการเข้าสังคม คือความสนุกที่มาจากการมีปฏิสัมพันธ์กับผู้อื่น 
การสื่อสาร การแข่งขัน เช่น ความสนุกจากการเล่าเรื่องราว 
ความสนุกจากการเล่นเกมออนไลน์
\subsubsection{Mental Fun}
ความสนุกทางด้านจิตใจ ไม่ว่าจะเป็นขณะที่คิดวางแผน 
หรือสามารถทำได้ตามแผนจนชนะ ทำให้ผู้เล่นรู้สึกดีต่อตนเอง 
เกิดความมั่นใจ และรู้สึกสนุก

\subsection{Maslow’s Hierarchy of Needs}

\begin{figure}[!h]\centering
\includegraphics[width=10cm]{./imgs/2-1.png}
\caption{ลำดับขั้นความต้องการของมาสโลว์}\label{fig:2-1}
\small [ที่มา : \url{https://droidinterface.com/pages/people_helping_animals_top_scale_human_needs.php}]
\end{figure}

แอบราฮัม มาสโลว์ (Abraham Maslow) อธิบายถึงพฤติกรรมของมนุษย์ว่าจะมีความต้องการ
เป็นระดับต่าง ๆ เรียงลำดับจากความต้องการระดับพื้นฐานไปยังระดับสูงสุด 8 ระดับ ดังนี้

\subsubsection{Physiological Needs}
\subsubsection{Physiological Needs}
\subsubsection{Physiological Needs}
\subsubsection{Physiological Needs}
\subsubsection{Physiological Needs}
\subsubsection{Physiological Needs}
\subsubsection{Physiological Needs}




% % Can define this in the preamble..
% You can place the figure and refer to it as รูปที่~\ref{fig:model2}.
% The figure and table numbering will be run and updated automatically when you add/remove tables/figures from the document.

\begin{figure}[!h]\centering
\setlength{\fboxrule}{0.2mm} % can define this in the preamble
\setlength{\fboxsep}{1cm}
\fbox{\includegraphics[width=5cm]{./imgs/logo02.jpg}}
\caption{The network model}\label{fig:model2}
\end{figure}

 
\subsection{อัลกอริทึม I}
Add more subsections as you want.


\section{เครื่องมือที่ใช้ในการพัฒนา}

%%%%%%%%%%%%%%%%%%%%%%%%%%%%%%%%%%%%%%%%%%%%%%%%%%%%%55
%%%%%%%%%%%%%%%%%%%%%%%%%%%%%%%%%%%%%%%%%%%%%%%%%%%%%
%%%%%%%%%%%%%%%%%%%%%%%%%%%%%%%%%%%%%%%%%%%%%%%%%%%%%
\chapter{วิธีการดำเนินงาน}

Explain the design (how you plan to implement your work) of your project. Adjust the section titles below to suit the types of your work. Detailed physical design like circuits and source codes should be placed in the appendix.

\section{ข้อกำหนดและความต้องการของระบบ}

\section{สถาปัตยกรรมระบบ}

\begin{table}[!h]
\centering
\caption{test table x1}\label{tbl:symbols}
\begin{tabular}{@{}p{0.07\textwidth}|p{0.7\textwidth}p{0.1\textwidth}}\hline
\multicolumn{2}{l}{\textbf{SYMBOL}}  & \textbf{UNIT} \\ \hline 
$\alpha$ & Test variable\hfill & m$^2$ \\
$\lambda$ & Interarrival rate\hfill &  jobs/second\\
$\mu$ & Service rate\hfill & jobs/second \\ \hline
\end{tabular}
%\begin{tabular}{c|c} \hline
% $\alpha$ & $\beta$ \\ \hline
% $\delta$ & $\mu$ \\ \hline
%\end{tabular}
\end{table}



\section{Hardware Module 1}
\subsection{Component 1}
\subsection{Logical Circuit Diagram}

\section{Hardware Module 2}
\subsection{Component 1}
\subsection{Component 2}

\section{Path Finding Algorithm}

\section{Database Design}

\section{UML Design}

\section{GUI Design}

\section{การออกแบบการทดลอง}
\subsection{ตัวชี้วัดและปัจจัยที่ศึกษา}
\subsection{รูปแบบการเก็บข้อมูล}




%%%%%%%%%%%%%%%%%%%%%%%%%%%%%%%%%%%%%%%%%%%%%%%%%%%%%%%%%%%%%%
%%%%%%%%%%%%%%%%%%%% Experiments %%%%%%%%%%%%%%%%%%%%%%%%%%%%%
%%%%%%%%%%%%%%%%%%%%%%%%%%%%%%%%%%%%%%%%%%%%%%%%%%%%%%%%%%%%%%%
\chapter{ผลการดำเนินงาน}

You can title this chapter as \textbf{Preliminary Results} ผลการดำเนินงานเบื้องต้น or \textbf{Work Progress} ความก้าวหน้าโครงงาน for the progress reports. Present implementation or experimental results here and discuss them.
ใส่เฉพาะหัวข้อที่เกี่ยวข้องกับงานที่ทำ 

\section{ประสิทฺธิภาพการทำงานของระบบ} 
\section{ความพึงพอใจการใช้งาน}
\section{การวิเคราะห์ข้อมูลและผลการทดลอง}

%%%%%%%%%%%%%%%%%%%%%%%%%%%%%%%%%%%%%%%%%%%%%%%%%%%%%%%%%%%%%%%
%%%%%%%%%%%%%%%%%%%% Conclusions %%%%%%%%%%%%%%%%%%%%%%%%%%%%%
%%%%%%%%%%%%%%%%%%%%%%%%%%%%%%%%%%%%%%%%%%%%%%%%%%%%%%%%%%%%%%%
\chapter{บทสรุป}

This chapter is optional for proposal and progress reports but 
is required for the final report.

\section{สรุปผลโครงงาน}
สรุปว่าโครงงานบรรลุตามวัตถุประสงค์ที่ตั้งไว้หรือไม่ อย่างไร 

\section{ปัญหาที่พบและการแก้ไข}
State your problems and how you fixed them.

\section{ข้อจำกัดและข้อเสนอแนะ}
ข้อจำกัดของโครงงาน What could be done in the future to make your projects better.

%//TODO: บรรณานุกรม
%%%%%%%%%%%%%%%%%%%%%%%%%%%%%%%%%%%%%%%%%%%%%%%%%%%%%%%%%%%%%%%
%%%%%%%%%%%%%%%%%%%% Bibliography %%%%%%%%%%%%%%%%%%%%%%%%%%%%%
%%%%%%%%%%%%%%%%%%%%%%%%%%%%%%%%%%%%%%%%%%%%%%%%%%%%%%%%%%%%%%%

%%%% Comment this in your report to show only references you have
%%%% cited. Otherwise, all the references below will be shown.
%\nocite{*}
%% Use the kmutt.bst for bibtex bibliography style 
%% You must have cpe.bib and string.bib in your current directory.
%% You may go to file .bbl to manually edit the bib items.

\makeatletter
\g@addto@macro{\UrlBreaks}{\UrlOrds}
\makeatother

\bibliographystyle{kmutt}
\bibliography{string,cpe}

%%%%%%%%%%%%%%%%%%%%%%%%%%%%%%%%%%%%%%%%%%%%%%%%%%%%%%%%%%%%%%%
%%%%%%%%%%%%%%%%%%%%%%%% Appendix %%%%%%%%%%%%%%%%%%%%%%%%%%%%%
%%%%%%%%%%%%%%%%%%%%%%%%%%%%%%%%%%%%%%%%%%%%%%%%%%%%%%%%%%%%%%%
\appendix{ชื่อภาคผนวกที่ 1}
\noindent{\large\bf ใส่หัวข้อตามความเหมาะสม} \\

This is where you put hardware circuit diagrams, detailed experimental data in tables or source codes, etc.. \\ \bigskip



This appendix describes two static allocation methods for fGn (or fBm)
traffic. Here, $\lambda$ and $C$ are respectively the traffic arrival
rate and the service rate per dimensionless time step. Their unit are
converted to a physical time unit by multiplying the step size
$\Delta$. For a fBm self-similar traffic source,
Norros~\cite{norros95} provides its EB as
\begin{equation}\label{eq:norros}
  C = \lambda + (\kappa(H)\sqrt{-2\ln\epsilon})^{1/H}a^{1/(2H)}x^{-(1-H)/H}\lambda^{1/(2H)}
\end{equation}
where $\kappa(H) = H^H(1-H)^{(1-H)}$. Simplicity in the calculation is
the attractive feature of (\ref{eq:norros}).

The MVA technique developed in~\cite{kim01} so far provides the most
accurate estimation of the loss probability compared to previous
bandwidth allocation techniques according to simulation results.
Consider a discrete-time queueing system with constant service rate
$C$ and input process $\lambda_n$ with $\mathbb{E}\{\lambda_n\} =
\lambda$ and $\mathrm{Var}\{\lambda_n\} = \sigma^2$.  Define $X_n \equiv
\sum_{k=1}^n \lambda_k - Cn$.  The loss probability due to the MVA
approach is given by
\begin{equation}\label{eq:loss_mva}
  \varepsilon \approx \alpha e^{-m_x/2}
\end{equation}
where
\begin{equation}\label{eq:mx}
m_x = \min_{n \geq 0} \frac{((C-\lambda)n + B)^2}{\mathrm{Var}\{X_n\}} =
\frac{((C-\lambda)n^\ast + B)^2}{\mathrm{Var}\{X_{n^{\ast}}\}}
\end{equation} 
and 
\begin{equation}\label{eq:alpha}
  \alpha =
  \frac{1}{\lambda\sqrt{2\pi\sigma^2}}\exp\left(\frac{(C-\lambda)^2}{2\sigma^2}\right)
  \int_C^\infty (r-C)\exp\left(\frac{(r-\lambda)^2}{2\sigma^2}\right)\, dr
\end{equation}
For a given $\varepsilon$, we numerically solve for $C$ that satisfies
(\ref{eq:loss_mva}). Any search algorithm can be used to do the task.
Here, the bisection method is used.  

Next, we show how $\mathrm{Var}\{X_n\}$ can be determined.  Let
$C_{\lambda}(l)$ be the autocovariance function of $\lambda_n$.  The
MVA technique basically approximates the input process $\lambda_n$
with a Gaussian process, which allows $\mathrm{Var}\{X_n\}$ to be
represented by the autocovariance function.  In particular, the
variance of $X_n$ can be expressed in terms of $C_{\lambda}(l)$ as
\begin{equation}
  \mathrm{Var}\{X_n\} = nC_{\lambda}(0) + 2\sum_{l=1}^{n-1} (n-l)C_{\lambda}(l)
\end{equation} 
Therefore, $C_{\lambda}(l)$ must be known in the MVA technique, either
by assuming specific traffic models or by off-line analysis in case of
traces.  In most practical situations, $C_{\lambda}(l)$ will not be
known in advance, and an on-line measurement algorithm developed
in~\cite{eun01} is required to jointly determine both $n^\ast$ and
$m_x$. For fGn traffic, $\mathrm{Var}\{X_n\}$ is equal to $\sigma^2
n^{2H}$, where $\sigma^2 = \mathrm{Var}\{\lambda_n\}$, and we can find
the $n^\ast$ that minimizes (\ref{eq:mx}) directly. Although $\lambda$
can be easily measured, it is not the case for $\sigma^2$ and $H$.
Consequently, the MVA technique suffers from the need of prior
knowledge traffic parameters.

%%%%%%%%%%%%%%%%%%%%%%%
% inregist %
% test for conflict check 

%%%%%%%%%%%%%%%%%%%%%%%%%%%%%%%%%%%%%%%%%%%%%%%%%%%%%%%%%%
%%%%%%%%%%%%%%% The 2nd appendix %%%%%%%%%%%%%%%%%%%%%%%%%%
%%%%%%%%%%%%%%%%%%%%%%%%%%%%%%%%%%%%%%%%%%%%%%%%%%%%%%%%%%
\appendix{ชื่อภาคผนวกที่ 2}
\noindent{\large\bf ใส่หัวข้อตามความเหมาะสม} \\

Next, we show how $\mathrm{Var}\{X_n\}$ can be determined.  Let
$C_{\lambda}(l)$ be the autocovariance function of $\lambda_n$.  The
MVA technique basically approximates the input process $\lambda_n$
with a Gaussian process, which allows $\mathrm{Var}\{X_n\}$ to be
represented by the autocovariance function.  In particular, the
variance of $X_n$ can be expressed in terms of $C_{\lambda}(l)$ as
\begin{equation}
  \mathrm{Var}\{X_n\} = nC_{\lambda}(0) + 2\sum_{l=1}^{n-1} (n-l)C_{\lambda}(l)
\end{equation} 

\noindent{\large\bf Add more topic as you need} \\

Therefore, $C_{\lambda}(l)$ must be known in the MVA technique, either
by assuming specific traffic models or by off-line analysis in case of
traces.  In most practical situations, $C_{\lambda}(l)$ will not be
known in advance, and an on-line measurement algorithm developed
in~\cite{eun01} is required to jointly determine both $n^\ast$ and
$m_x$. For fGn traffic, $\mathrm{Var}\{X_n\}$ is equal to $\sigma^2
n^{2H}$, where $\sigma^2 = \mathrm{Var}\{\lambda_n\}$, and we can find
the $n^\ast$ that minimizes (\ref{eq:mx}) directly. Although $\lambda$
can be easily measured, it is not the case for $\sigma^2$ and $H$.
Consequently, the MVA technique suffers from the need of prior
knowledge traffic parameters. 





\end{document}
